\chapter{Introduction}\label{introduction}

% Motivation
\section{Motivation}\label{intro:motivation}

Based on a recent study, 68 percent of US adults said that they at least occasionally get news on social media; however, 57 percent of those people expect the news to be largely inaccurate \cite{matsa2018news}. The facts presented outlines a fundamental problem with the social media platform; verifying the truthfulness of the articles. Recently, the term "fake news" has soared in popularity across various social media, news industry, and research alike. Jaster et al. define the term as \cite{jasterfake}

\enquote{Fake news is either false or misleading information that is propagated with either the intention to deceive or an utter disregard for the truth.}

\noindent
which mostly covers what is meant when the term "fake news" is used. A study has shown that fake news viewing impacts political attitudes, enhances the feelings of inefficacy, alienation, and cynicism toward politicians \cite{balmas2014fake}. In a similar manner, others have pointed out that fake news could impact the ability of people to accept a truthful news by confusing them with false stories \footnote{\url{https://www.nytimes.com/2016/11/28/opinion/fake-news-and-the-internet-shell-game.html?\%20r=0}}.

During the United States presidential election of 2016, it has been shown that the most popular fake news stories were shared more on Facebook than the most popular mainstream news stories\cite{silverman2016analysis}. It was also reported that 75 percent of people who saw the fake news stories at that time said that they believed them\cite{silverman2016most}. Combining the previous facts with the fact that the most popular fake news stories were in favor of Donald Trump over Hillary Clinton\cite{silverman2016analysis}, a number of commentators proposed that the outcome of the election might have been different were it not for the influence of fake news \cite{parkinson2016click, read2016donald, dewey2016facebook}

In order to control the effects that a fake news might have it is essential to develop systems that can identify fake news as early as possible. However, automatic detection of fake news poses several problems; it might have been intentionally written to mislead readers, which make it difficult to detect based on content only, on the other hand using auxiliary information such as knowledge base and social engagements actually leads to another problem of verifying the quality of the data itself\cite{shu2017fake}. It is also impractical for the reviewers to analyze every single article, statement or message that is found in the social media and decide whether it is a fake news. Due to these reasons, a number of researchers have even proposed that a better compromise could be a semi automated system where an algorithm would augment a human reviewer to identify a fake news \cite{conroy2015automatic, chen2015news, wiegand2016veracity}.

One key observation that can be fundamental in combating fake news is that we do not need to analyze every single article, message or statement and evaluate its truthfulness rating. There are articles that describe topics such as infotainment, personal news, beauty tips, etc. which might not have severe impact even if they were fake. Such articles can be safely excluded from the evaluation step or can be verified as needed. 
%Once we do that what remains is a collection of news articles which should be evaluated for the truthfulness rating. In other words, t
There are somewhat important articles that needs to be verified and there are others which may not matter even if they are fake.
For example, consider the following statement: "I watched the Black Panther movie in the weekend. Fantastic effects!". Detecting whether the person may have watched the movie or not, or whether the movie has fantastic effects may not be considered as critical. However, a statement such as "There has been a shooting at the mall" needs to be verified. 

% Research Problem
\section{Research Problem} \label{intro:research}
The concept of separating news into somewhat important news which needs to be verified and other slightly unimportant news could be in a way compared to the dichotomy of news into hard and soft. There are several definitions for hard and soft news \cite{reinemann2012hard, shoemaker2012news} but the one by Reinemann et al., 2012 is simple and stated as \cite{reinemann2012hard},

\enquote{Hard news is defined as reports about politics, public administration, the economy, science, technology and related topics. Soft news is defined as reports about celebrities, human interest, sport and other entertainment-centred stories.}

The categorization of text into hard news and soft news is inadequate for the purpose of identifying important article from unimportant ones because there might be articles in hard news which are not that important and soft news might also have articles which are important. So a new term "critical news" is introduced which is defined as follows,

\enquote{A text is labeled as critical if it affects significant number of people, changes the routines of daily life, and needs verification on the information presented.}

Any article whose content affects significant number of people, changes their routines of daily life, and needs to be verified could be labelled as critical news and all other articles which does not meet those criteria should be labelled as a non critical news. 

Many research studies aim at detecting fake news. Although there have been already datasets for studying fake news, there is no dataset available for evaluation of detecting critical news as mentioned in this thesis. To study this problem, we work on the following problems:

\begin{enumerate}
    \item \textit{Building a new dataset for critical news detection.}
    \item \textit{Studying existing fake news detection methods for detecting critical news.}
    \item \textit{Selecting features from fake news detection methods and evaluating a number of classifier models for critical news detection.}
\end{enumerate}

Especially, building the dataset for critical news detection was complex. Finding ground-truth or reference messages for labeling was important for this study. Once these messages are labeled based on the definition of critical news, we built a number of classifiers.
Therefore, the eventual goal is to be able to construct a binary classifier that is able to separate a critical news from a non critical news based on it's content. Once these news statements are detected as critical, they can be evaluated whether they are fake or not.


% Thesis Organization
\section{Thesis Organization} \label{intro:organization}
The remainder of the thesis is organized as follows. Chapter 2 covers brief background for the text classification problems and provides related work on research studies that are related to this thesis. Chapter 3 presents how the dataset is built and the methodology that was used to train, validate and test various classifier models. Chapter 4  provides the results and analysis of the experiments. And, Chapter 5 concludes the thesis with the possible future directions that could be explored.