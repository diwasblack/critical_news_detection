\chapter{Introduction}\label{introduction}

% Motivation
\section{Motivation}\label{intro:motivation}
Based on a recent study, 68 percent of US adults said they at least occasionally get news on social media; however, 57 percent of those people expect the news to be largely inaccurate \cite{matsa2018news}. These facts portray a significant problem with the social media platforms: verifying the truthfulness of the articles posted on them. Recently, the term "fake news" has soared in popularity across various social media, news industry, and research papers alike. Jaster et al. define the term as \cite{gelfert2018fake}

\enquote{Fake news is the deliberate presentation of (typically) false or misleading claims as news, where the claims are misleading by design.}

\noindent
A study has shown that fake news viewing impacts political attitudes, enhances the feelings of inefficacy, alienation, and cynicism toward politicians \cite{balmas2014fake}. And, others have pointed out that fake news could impact the ability of people to accept truthful news by confusing them with false stories \footnote{\url{https://www.nytimes.com/2016/11/28/opinion/fake-news-and-the-internet-shell-game.html?\%20r=0}}.

During the United States presidential election of 2016, fake news propagated far more rapidly on Facebook than did real news from mainstream media\cite{silverman2016analysis}, and it was also reported that 75 percent of people who saw the fake news stories at that time said that they believed them\cite{silverman2016most}. There were fake news for both candidates despite one candidate had more fake news in favor than the other candidate \cite{silverman2016analysis}. This led to a number of commentators proposing that the fake news influenced the election  \cite{parkinson2016click, read2016donald, dewey2016facebook}

To control the effects that fake news might have, it is essential to develop systems that can identify fake news as early as possible. However, automatic detection of fake news poses several problems. Firstly, in order to achieve its purpose of misleading readers, a fake news story will try its best to appear genuine. Therefore, based solely on content, it is difficult to classify the story as fake. Secondly, using auxiliary information such as knowledge base and social engagements actually leads to another problem of verifying the quality of the data itself\cite{shu2017fake}. It is also impractical for the reviewers to analyze every single article, statement or message that is found in social media and decide whether it is a fake news. Due to these reasons, a number of researchers have even proposed that a better compromise could be a semi-automated system where an algorithm would augment a human reviewer to identify fake news \cite{conroy2015automatic, chen2015news, wiegand2016veracity}.

One key observation that can be fundamental in combating fake news is that we do not need to analyze every single article, message or statement. There are articles that describe topics such as infotainment, personal news, beauty tips, etc. which might not have severe impact even if they were fake. Such articles can be excluded from the verification step or can be verified as needed. For example, consider the following statement: "I watched the Black Panther movie in the weekend. Fantastic effects!". Detecting whether the person may have watched the movie or not, or whether the movie has fantastic effects may not be considered as important. However, a statement such as "There has been a shooting at the mall" needs to be verified.
To summarize, we could say that there are somewhat important articles that needs to be verified and there are others which may be excluded from the verification step.

% Research Problem
\section{Research Problem} \label{intro:research}
The concept of separating news into somewhat important news which needs to be verified and other slightly unimportant news could be in a way compared to the dichotomy of news into hard and soft. There are several definitions for hard and soft news \cite{reinemann2012hard, shoemaker2012news} but the one by Reinemann et al., 2012 is simple \cite{reinemann2012hard}: 

\enquote{Hard news is defined as reports about politics, public administration, the economy, science, technology and related topics. Soft news is defined as reports about celebrities, human interest, sport and other entertainment-centred stories.}

The categorization of text into hard news and soft news is inadequate for the purpose of identifying important article from unimportant ones because there might be articles in hard news which are not that important and soft news might also have articles which are important. So a new labelling "significant" is introduced which is defined as follows:

\enquote{A text is labeled as significant if it affects a large number of people, changes the routines of daily life, and needs verification on the information presented.}

Any article whose content affects large number of people, changes their routines of daily life, and needs to be verified could be labelled as significant and all other articles which does not meet those criteria should be labelled as a non-significant. 

Many research studies aim at detecting fake news. Although there already exist datasets for studying fake news, there is no dataset available for detecting significant news as mentioned in this thesis. To study this problem, we work on the following problems:

\begin{enumerate}
    \item Building a new dataset for significant news detection.
    \item Studying existing fake news detection methods for detecting significant news.
    \item Selecting features from fake news detection methods and evaluating a number of classifier models for significant news detection.
\end{enumerate}

Especially, building the dataset for significant news detection was complex. Finding ground-truth or reference messages for labeling was important for this study. Once these messages are labeled based on the definition of significant news, we built a number of classifiers. Therefore, the eventual goal is to be able to construct a binary classifier that is able to separate significant news from non-significant news based on its content. Once these news statements are detected as significant, they can be further evaluated to determine whether they are fake or not.

% Thesis Organization
\section{Thesis Organization} \label{intro:organization}
The remainder of the thesis is organized as follows. Chapter 2 covers brief background for the text classification problems and provides research studies that are related to this thesis. Chapter 3 presents how the dataset is built and the methodology that was used to train, validate and test various classifier models. Chapter 4  provides the results and analysis of the experiments. And finally, Chapter 5 concludes the thesis with possible future directions that could be explored.