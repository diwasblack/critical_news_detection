\chapter{Introduction}\label{introduction}

% Motivation
\section{Motivation}\label{intro:motivation}

Based on a recent study, 68 percent of US adults said that they at least occasionally get news on social media \cite{matsa2018news}. However, 57 percent of those people expect the news to be largely inaccurate. The facts presented outline a fundamental problem with the social media platform; verifying the truthfulness of the articles. Recently, the term "fake news" has soared in popularity across various social media, news industry, and research alike. Jaster et al. define the term as \cite{jasterfake},

\enquote{Fake news is either false or misleading information that is propagated with either the intention to deceive or an utter disregard for the truth.}

which mostly covers what is meant when the term "fake news" is used. A study has shown that fake news viewing impacts political attitudes, enhances the feelings of inefficacy, alienation, and cynicism toward politicians \cite{balmas2014fake}. In a similar manner, others have pointed out that fake news could impact the ability of people to accept a truthful news by confusing them with false stories \footnote{\url{https://www.nytimes.com/2016/11/28/opinion/fake-news-and-the-internet-shell-game.html?\%20r=0}}.

During the United States presidential election of 2016, it has been shown that the most popular fake news stories were more shared on Facebook than the most popular mainstream news stories\cite{silverman2016analysis}. It was also reported that 75 percent of people who saw the fake news stories at that time said that they believed them\cite{silverman2016most}. Combining the previous facts with the fact that the most popular fake news stories were in favor of Donald Trump over Hillary Clinton\cite{silverman2016analysis}, a number of commentators proposed that Donald Trump would not have been elected president were it not for the influence of fake news \cite{parkinson2016click} \cite{read2016donald} \cite{dewey2016facebook}.

In order to control the effects that a fake news might have it is essential to develop systems that can identify fake news as early as possible. Automatic detection of fake news poses several problems; it might have been intentionally written to mislead readers, which make it difficult to detect based on content only, on the other hand using auxiliary information such as knowledge base and social engagements actually leads to another problem of verifying the quality of the data itself\cite{shu2017fake}. It is also impractical for the reviewers to analyze every single article that is found in the social media and decide whether it is a fake news. Due to these reasons, a number of researchers have even proposed that a better compromise could be a semi automated system where an algorithm would augment a human reviewer to identify a fake news \cite{conroy2015automatic} \cite{chen2015news} \cite{wiegand2016veracity}. 

One key observation that can be fundamental in combating fake news is that we do not need to analyze every single article and evaluate it's truthfulness rating. An article might describe topics such as infotainment, personal news, health and beauty tips, etc which might not have severe impact even if there were fake. Such articles can be safely discarded from the evaluation step. Once we do that what remains is a collection of news articles which should be evaluated for the truthfulness rating. In other words, there are somewhat important articles that needs to be verified and there are others which does not matter even if they are fake.

% Research Problem
\section{Research Problem} \label{intro:research}
The concept of separating news into somewhat important news which needs to be verified and other slightly unimportant news could in way be compared to the dichotomy of news into hard and soft. Various people have proposed different definition for hard and soft news\cite{reinemann2012hard}\cite{shoemaker2012news}. Reinemann et al., 2012 define hard and soft news as \cite{reinemann2012hard},

\enquote{Hard news is defined as reports about politics, public administration, the economy, science, technology and related topics. Soft news is defined as reports about celebrities, human interest, sport and other entertainment-centred stories.}

The categorization of news into hard and soft is inflexible and is inadequate for the purpose of identifying important article from unimportant ones. So a new term "critical news" is introduced which is defined as follows,

\enquote{A text is labeled as critical if it affects significant number of people, changes the routines of daily life, and needs verification on the information presented.}

Any article whose content satisfied the pre-conditions could be labelled as a critical news. So, the goal is to construct a classifier that is able to separate a critical news from a non critical news based on the pre-conditions.

% Thesis Organization
\section{Thesis Organization} \label{intro:organization}
The remainder of the thesis is organized as follows. Chapter 2 covers the necessary background for the text classification problems and provides with the research works that are related to this thesis. Chapter 3 presents the methodology that was used to train, validate and test various classifier models. Which is followed by Chapter 4 which provides the results and analysis of the experiments. And, Chapter 5 concludes the thesis with the possible future directions that could be explored.