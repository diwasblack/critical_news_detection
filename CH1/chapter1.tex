\chapter{Introduction}\label{introduction}

\section{Motivation}\label{intro:motivation}
According to the authors of "What is fake news?"\cite{jasterfake}, 

\enquote{Fake news is either false or misleading information that is propagated with either the intention to deceive or an utter disregard for the truth.}

Based on a study of 2016, for 62 percent of US adults their source for news was social media platform such as Facebook\cite{gottfried2016news}. And it also has been shown that the most popular fake news stories
were more shared on Facebook than the most popular mainstream news stories\cite{silverman2016analysis} and many people who saw fake news stories reported that they believed them\cite{silverman2016most}. Such widespread of fake 
news can have severe impact on people and the society as a whole.  

During the United States presidential election of 2016, the most popular fake news stories were in favor of Donald Trump over Hillary Clinton\cite{silverman2016analysis}, combining the facts presented above, a number of commentators proposed that Donald Trump would not have been elected president were it not for the influence of fake news\cite{parkinson2016click}\cite{read2016donald}\cite{dewey2016facebook}.

Social media platforms contains very large volume of content which might contain fake news as well so it is necessary to detect a fake news beforehand in order to reduce the negative effects it might cause.

\section{Research Problem} \label{intro:research}
Automatic detection of fake news poses several problems; it might have been intentionally written to mislead readers, which make it nontrivial to detect simply based on content, exploiting auxiliary information such as knowledge base and user social engagements actually leads to another problem of verifying the quality of the data itself\cite{shu2017fake}.

Due to the high velocity and veracity of the news found in the social media it would be better to build a semi-automated system\cite{wiegand2016veracity}.

Hard and soft news \cite{reinemann2012hard}

\enquote{Hard news is defined as reports about politics, public administration, the economy, science, technology and
related topics. Soft news is defined as reports about celebrities, human interest, sport and
other entertainment-centred stories.}

\section{Thesis Organization} \label{intro:organization}