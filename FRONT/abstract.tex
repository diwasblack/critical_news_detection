\chapter*{Abstract}
% the \makeabstract command creates the top portion of the abstract
% page ... must be issued before the abstract content
\makeabstract

%%%%%%%%%%% Your Abstract Text Goes after Here %%%%%%%%%%%%%%%%%%%%%%%

Protein crystallization screening is the process of evaluating reagents and chemicals with the objective of producing protein crystals. Many combinations of chemicals need to be experimented to obtain crystalline conditions. This thesis presents a novel way of identifying the various conditions necessary for a successful crystal growth by using a variation of the genetic algorithm which explores unexplored territories of the chemical search space, thereby increasing the probability of finding additional crystalline conditions.

Novelty search using genetic algorithm aims generating population that has a diverse set of individuals by exploring a large search space.
This thesis also analyzes if there is any redundancy in the evaluation of the fitness score for 
individuals. The fitness score calculation can be time consuming especially because the fitness function is complex for novelty search. 
Determining distinctive individuals may be an overhead for the algorithm due to a significant number of individuals in many populations.
%since distinctiveness could already be evaluated in the previous populations generated by the genetic algorithm. 
This thesis presents a method to reduce the redundancy by building dictionaries of the individuals already encountered.
%Even while exploring a larger area of search space, the genetic algorithm generates a 
%does not guarantee that the individuals in new generations will always be different from the individuals from previous generation. 
%Evaluation of fitness score of intermediate individuals can be expensive especially when the fitness function is complex. This thesis also discusses on how a simple dictionary data structure can be used to reduce the redundancy involved in evaluating the fitness score of the individuals already encountered by the algorithm. 


The evaluation of our method on protein crystallization yields 6 common crystalline conditions with associative experimental design(AED) and 9 common crystalline conditions with GenScreen for the protein AbIPPase. Wet lab experiments for the protein ConA show that our method could produce 96 crystalline conditions utilizing more number of distinct precipitants, distinct salts, and distinct buffers than AED and GenScreen. 
These results show that our method can effectively produce additional crystalline conditions by exploring a larger part of the chemical search space. Furthermore, by avoiding redundant computations and using dictionary of individuals, we were able to decrease the runtime by a factor of 10 in our experiments.% of this application by a factor of 27.

%reducing the overhead associated with the fitness score calculation.




%%%%%%%%%%%%% Your Abstract Text should be before Here %%%%%%%%%%%%%%

% the abstractsig command creates the signature spaces after the
% abstract, and therefore, must be issued after the abstract.
\abstractsig
