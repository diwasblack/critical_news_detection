\chapter*{Abstract}
% the \makeabstract command creates the top portion of the abstract
% page ... must be issued before the abstract content
\makeabstract

%%%%%%%%%%% Your Abstract Text Goes after Here %%%%%%%%%%%%%%%%%%%%%%%
Social media platforms nowadays have a large number of fake or false news which have been misleading and negatively impacting viewers. In order to combat the problem, being able to differentiate important news stories which need to be verified from unimportant news stories which need not, would be a decent starting point. The important news articles are the ones which are called "critical" and the rest are "non-critical". This thesis then explores if it is possible to construct a classifier for detecting critical news articles.

A dataset containing 1548 critical and 595 non-critical articles was prepared by manually labelling the posts obtained from Twitter. Various classifiers -- including Logistic Regression, Support Vector Machine, Random Forests, and Neural Network -- were trained on the dataset. They each achieved an accuracy greater than 90 percent, with the Neural Network model achieving the highest accuracy of 93.654 percent. This indicates that it is in fact possible to build fairly accurate classifiers for critical news detection. This thesis also describes a few possible future directions that could be explored for further improving the performance of critical news detection.

%%%%%%%%%%%%% Your Abstract Text should be before Here %%%%%%%%%%%%%%
\clearpage

% the abstractsig command creates the signature spaces after the
% abstract, and therefore, must be issued after the abstract.
\abstractsig
