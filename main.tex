%   thesis.tex - UAH Master's Thesis Main TeX file
%
%   This is the main file for my thesis. It loads all content via
%   \include statements

%%% Making the DVI and PDF output the correct size %%%
%   Important - the default pagesize for DVI and PDF output in most LaTeX
%   distributions is a4 (210x297cm, or about 8.27×11.69 in), which won't work
%   for the UAH format.  So, you have to change the DVI and PDF output in your
%   LaTeX distribution.  In most distros, you need to change the following files:

%   (1.)  C:\texmf\dvipdfm\config\dvipdfmx.cfg
%   (2.)  C:\texmf\dvipdfm\config\config
%   (3.)  C:\texmf\pdftex\config\pdftex.cfg

%   For numbers (1.) and (2.), you will change where the files say "p a4" to
%   "p letter" (there should be some explanation in the files themselves, too).
%   For number (3.), you will change two lines of that file to read:

%   page_height 11 true in
%   page_width 8.5 true in

%   There are "model-thesislocal" versions of all the above files in this complete rar
%   package which 'should' override the a4 defaults - but these do not work in
%   some distributions, so editing the above files is usually necessary.
%   For more info, search google for "pdftex.cfg" and "dvipdfmx.cfg"
%%% (End DVI and PDF changes) %%%

%   Define the document class. The proper class is "book." The normal options
%   are "12pt" and "onesided". When printing out copies for purposes other than
%   submission to the Graduate School, the option "twoside" can be used to
%   format the output in a manner which looks good when printed duplex.

\documentclass[12pt,oneside,letterpaper]{book}
%\documentclass[12pt,twoside,letterpaper]{book}

%   Load the packages necessary for the thesis. The only required packages are
%   "uahdis", which loads the UAH Dissertation Style, and "chngpage" which is
%   used by "uahdis". The other packages are optional, and should be loaded only
%   if you have a need for the functionality they provide. Documentation on each is
%   available at CTAN.org. The basic purpose of each are described below:
%   units---provides a convenient mechanism for writing quantities with units.
%   bm---provides a convenient mechanism for producing bold symbols in math mode
%   hhline---improved lines and borders in tables
%   rotating---allows landscape oriented pages in a portrait oriented document
%   verbatim---prints text files verbatim. useful for computer code
%   amsfonts---provides additional typefaces
%   amssymb---provides additional mathematical symbols
%   xspace---properly adjusts space after \newcommands which expand to text
%   booktabs---creates traditional scientific tables - no vertical lines, very
%   few horizontal lines, with varying thickness (not meant to be used with hline)

%   Excellent descriptions of all LaTeX packages can be found here:
%   http://www.tug.org/tex-archive/help/Catalogue/index.html

\usepackage[]{uahdis,chngpage,units,bm,hhline,rotating,verbatim,amsfonts,amssymb,xspace}
\usepackage[]{indentfirst,layouts, booktabs, enumerate, gensymb}

%My Packages
\usepackage[hyphens]{url}
\usepackage[none]{hyphenat}
\usepackage{amsmath}
% \usepackage{hyperref}
\usepackage{array, graphicx}
\usepackage{multirow}
\usepackage{microtype}
\usepackage{float}
\usepackage{placeins}
\usepackage{csquotes}

%\usepackage{url} % hyperref works too
%\urlstyle{same}  % (sf also works, for something more subtle than tt)

%   Load the tex file which contains all local \newcommands and \newenvironments

%%%
%%% Environments
%%%

\newsavebox{\speaker}
\newenvironment{chapterquote}[1]
    {\begin{flushright}\begin{minipage}{3 in}\sbox{\speaker}{#1}\itshape\singlespace}
    {\begin{flushright}---\usebox{\speaker}\end{flushright}\end{minipage}\end{flushright}}

% Note: chapterquote works (and looks) best when the chapter begins with a \section{} ...
% If you hadn't planned on beginning the chapter with a \section, try \section{Overview} :-)
% otherwise, some \vspace{} might be necessary, but that won't be consistent throughout the thesis

%%%
%%%  Functions---Commands that take parameters
%%%
\newcommand{\cfig}[3]{\centering\includegraphics[keepaspectratio=true,width=#3in]{./CH#1/EPSFDocs/#2}}
\newcommand{\acro}[1]{\textsc{#1}}
\newcommand{\sci}[2]{\ensuremath{#1 \!  \times \!  10^{#2}}}
\newcommand{\vect}[1]{\boldsymbol{\mathbf{#1}}}
\newcommand{\unitvec}[1]{\vect{\hat{#1}}}
\newcommand{\threebythree}[9]{\renewcommand{\arraystretch}{0.75}\begin{vmatrix}#1&#2&#3\\#4&#5&#6\\#7&#8&#9\end{vmatrix}\renewcommand{\arraystretch}{1.0}}
\newcommand{\mysci}[2]{\ensuremath{#1 \!  \times \!  10^{#2}}}
\newcommand{\myprop}[5]{g(#1,#2;#3,#4;#5)}
\newcommand{\myint}[3]{\int_{#1}^{#2}#3}
\newcommand{\myintinfinf}[1]{\myint{-\infty}{\infty}{#1}}
\newcommand{\minus}{\scalebox{0.75}[1.0]{$-$}}


%%%
%%%  New Math Operators
%%%

\DeclareMathOperator{\polylog}{Li}


%%%
%%%  Symbols---Commands that are shorthand
%%%
\newcommand{\rf}{\acro{rf}\xspace}
\newcommand{\dc}{\acro{dc}\xspace}
\newcommand{\ac}{\acro{ac}\xspace}
\newcommand{\ccd}{\acro{ccd}\xspace}
\newcommand{\mach}{\acro{mach2}\xspace}
\newcommand{\mgmhd}{\acro{mgmhd}\xspace}
\newcommand{\cea}{\acro{cea}\xspace}
\newcommand{\half}{\nicefrac{1}{2}\,}
\newcommand{\xx}{\ensuremath{\unitvec{x}}\xspace}
\newcommand{\yy}{\ensuremath{\unitvec{y}}\xspace}
\newcommand{\zz}{\ensuremath{\unitvec{z}}\xspace}
\newcommand{\ttl}{\acro{ttl}\xspace}
\newcommand{\pmt}{\acro{pmt}\xspace}
\newcommand{\eg}{\textit{e.g.}\xspace}
\newcommand{\ie}{\textit{i.e.}\xspace}
\newcommand{\qmn}{\ensuremath{q^{mn}}\xspace}
\newcommand{\isp}{\ensuremath{I_{sp}}\xspace}
\newcommand{\wrt}{\textit{w.r.t.}\xspace}


%%%
%%%  Other stuff I've found - Evaluate for usefulness
%%%

%\newcommand{\quan}[2][]{\mbox{$#1\,\mathrm{#2}$}}
%\newcommand{\vv}[1]{\ensuremath{\boldsymbol{#1}}}
%\newcommand{\tempc}[1]{\quan[#1]{^{\circ}C}}
%\DeclareMathOperator{\sinc}{sinc}
%\newcommand{\comment}[1]{\marginpar{\Large \hfill \ddag}\textsf{#1}}
%\newcommand{\comment}[1]{}


%%% my definition added 

%   The next line requires the leading "%". It is only useful if you are using
%   WinEdt as your text editor. It allows WinEdt to collect bibliographic entries into
%   a pop-up table that you can summon when you are \cite-ing a source.

%GATHER{Bibliography.bib}

%   Define my name
\author{Diwas Sharma}

%   Define the title of my thesis (user upper and lower case)
\title{Detecting Fake News}

%   Define the year
\date{2018}

%   Define my department
\uahdepartment{Computer Science}

%   Define my advisor (no title - i.e., no "Dr.")
\uahadvisor{Ramazan Aygun}

%   Define my committee members (no titles)
\uahmema{Huaming Zhang}

\uahmemb{Vineetha Menon}

% MWT - UAH Thesis - only three committee members
% if more than three, uncomment lines below
% and appropriate lines in uahdis.sty

%\uahmemc{Fourth Committee Member}

%\uahmemd{Fifth Committee Member}

%MWT

%   Define my Department Chair (no title)
\uahdeptchair{Heggere S. Ranganath}

%   Define my College (do not write "College of")
\uahcollege{Science}

%   Define my College Dean (no title)
%   the tex for Dean Aunon is:  Jorge I. Au\~n\a'on
\uahcolldean{Sundar Christopher}

%   Define my Degree (i.e., Master of Science, Doctor of Philosophy)
\uahdegree{Master of Science in Computer Science}

%   Define my Program name
\uahprogram{Computer Science}

%   Shortened name for degree
%   type "master's" or "doctoral" (without quotes)
\uahdegreeshort{master's}

%   Define my Document Type
\uahdoctype{thesis}

%   Define my Graduate Dean
\uahgraddean{David Berkowitz}


%   Let's get started (Finally!)
\begin{document}

%   To view a layout of the margins of this document,
%   uncomment the below line (requires "layouts" package):
%\layout
%\tocdiagram\tocdesign

%   Here comes the stuff that goes before the main content
\frontmatter

\pagestyle{plain}

\maketitle

\copyrightpage

%MWT - UAH Approval Form
\approvalpage

%\makeabstract % Now issued in ./FRONT/abstract.tex

%   Include my abstract
\chapter*{Abstract}
% the \makeabstract command creates the top portion of the abstract
% page ... must be issued before the abstract content
\makeabstract

%%%%%%%%%%% Your Abstract Text Goes after Here %%%%%%%%%%%%%%%%%%%%%%%
Social media platforms nowadays have a large number of fake or false news which have been misleading and negatively impacting viewers. In order to combat the problem, being able to differentiate important news stories which need to be verified from unimportant news stories which need not, would be a decent starting point. In this thesis, we introduce "significant news" and define it as news that affects a large number of people, changes the routines of daily life, and needs verification on the information presented. This thesis then explores if it is possible to construct a classifier for detecting the significant news articles.

A dataset containing 1548 significant and 595 non-significant articles was prepared by manually labelling the posts obtained from Twitter. Various classifiers including logistic regression, support vector machine, random forest, and neural network -- were trained on the dataset. They each achieved an accuracy greater than 90 percent, with the neural network model achieving the highest accuracy of 93.654 percent. This indicates that it is in fact possible to build fairly accurate classifiers for detecting significant news. This thesis then describes a few possible future directions that could be explored for further improving the performance of significant news detection.

%%%%%%%%%%%%% Your Abstract Text should be before Here %%%%%%%%%%%%%%

% the abstractsig command creates the signature spaces after the
% abstract, and therefore, must be issued after the abstract.
\abstractsig


% Abstract signatures:
%\abstractsig % command now issued in abstract.tex

%   Include my acknowledgements
\chapter*{Acknowledgments}

I would like to express my very great appreciation to Dr. Ramazan Aygun for his guidance and support throughout this research.

Also, I would also like extend my thanks to my family and friends who supported me in my endeavors.


%   Make a Table of Contents
\tableofcontents

%   Make a List of Figures
\listoffigures

%   Make a List of Tables
\listoftables

%   Make a List of Symbols (Comment out if unwanted)
%\listofsymbols

\noindent Acronyms

\symboldefinition{API}{Application Program Interface}
\symboldefinition{CNN}{Convolution Neural Network}
\symboldefinition{IDF}{Inverse Document Frequency}
\symboldefinition{LSTM}{Long Short Term Memory}
\symboldefinition{NLP}{Natural Language Processing}
\symboldefinition{RBF}{Radial Basis Function}
\symboldefinition{SVM}{Support Vector Machine}
\symboldefinition{TF}{Term Frequency}
\symboldefinition{t-SNE}{T-distributed Stochastic Neighbor Embedding}
\symboldefinition{SGD}{Stochastic Gradient Descent}
\symboldefinition{VSM}{Vector Space Modeling}

%   Make a List of Abbreviations (Comment out if unwanted)
% \include{./FRONT/abbreviations}

%   Make a "Chapter" header in TOC, per UAH style
%   issue command after last frontmatter TOC entry
%   cannot come directly before first chapter \include
\addchapheadtotoc

%   Set my dedication (optional - comment out if unwanted)
%\dedication{}

%   Make Epigraph Page (Optional - comment out if unwanted)

\clearpage

\pagestyle{myheadings} \markright{}

%   Here comes the main content
\mainmatter

%   Include my chapters
%   LaTeX will look for your chapter files in the appropriate folders,
%   as addressed below.  LaTeX will look for a .tex file with the
%   same name as the name you give it below - i.e., for the CH1 folder,
%   LaTeX will look for introduction.tex (which is where you will type
%   all your chapter 1 stuff).  You can name your chapters/files whatever
%   you want - just make sure the names below match the names in the
%   folders.  Also, you can add or subtract chapters as you like -
%   just make sure that the address and filenames below match your
%   file structure.

 \chapter{Introduction}
\label{ch:background}

\section{Motivation} \label{intro:motivation}
Protein crystallization screening is the process of identifying experimental conditions suitable for the formation of large protein crystals. Some of the factors that affect crystal growth are salts, pH of buffer, temperature, molar concentration, etc \cite{IntroFactor,History,McPherson}. The goal of screening is to find a list of combinations of these factors that are likely to produce a successful outcome for proteins. In other words, the goal of screening is to develop crystalline conditions suitable for determining the structure of a protein. 

Many screening methods have been developed for building protein crystallization screens. Carter et al. \cite{carter1979protein} showed that incomplete factorial design could be used for finding successful crystalline conditions. With sparse matrix sampling \cite{jancarik1991sparse} and analyzing the chemical space suitable for crystalline conditions, commercial screens have been developed. While these commercial screens have produced successful results for some proteins, similar results have not been obtained for more complex proteins. However, the results of these commercial screens can still be analyzed for experimenting with new sets of conditions.

Analyzing the output of the initial set of experiments to suggest new conditions to be tested is challenging when there are few or no successful outcomes. When the data is highly skewed (i.e., the significant majority of trials yield non-crystals) \cite{Skewed}, the computational models such as the regression  or the   classification methods are unlikely to produce a reliable model. Despite this complication, neural networks have been shown to output crystalline conditions in the past \cite{delucas2003efficient}. 
Even with models that are built based on prior results that yielded crystalline conditions, the chemical space will not necessarily have been explored effectively, as there is a possibility that the new test conditions is similar to the original test conditions. It is important to come up with a set of algorithms that can explore the chemical space effectively. In this thesis, we explore the usage of genetic algorithms in suggesting novel conditions. 

\section{Genetic Algorithm} \label{intro:genetic}
Genetic algorithm  is an evolutionary algorithm inspired by the process of biological evolution. Like in nature, the evolution starts with a population of individuals, and iteratively, the genomes of the fittest individuals are propagated to create the next generation of individuals. There are many variants of genetic algorithms with possible modifications to its steps. Variations of genetic algorithms could be developed by changing the encoding of the chromosomes, or the fitness functions, or setting different mutation rates, etc.  

Genetic algorithm is generally used to solve optimization and search problems. It is very suitable for protein crystallization as its output is a population of cocktails (conditions). Acharya \cite{SamyamThesis} has shown  how genetic algorithms can be applied for protein crystallization screening. However, it is very likely of the genetic algorithm only exploring certain parts of the chemical search space, thereby missing cocktails that are further away from the original population. Usually, the mutation rate helps explore larger chemical space leading to  the generation of novel conditions.
%The major novelty is introduced by the mutations.

\section{Research Problem} \label{intro:research}
There is a problem in developing novel conditions from previous experiments. To build a model based on the analysis of previous experiments, and to expect new crystalline conditions different from the original set of conditions, would require obtaining new information from a stagnant data set.

This thesis deals with the problem of determining a combination of conditions that can produce large protein crystals. For producing these large novel crystals, we propose the use of a variant of the genetic algorithm that can explore unchartered territories of the chemical search space.
To achieve this, we will modify the genetic algorithm such that it uses a fitness function that can directly consider the novelty of an individual (i.e., the most novel or different individuals are rewarded positively \cite{Novelty}). 
We will then propose an appropriate distance metric to measure the novelty. 
In using a fitness function that solely depends on novelty, the algorithm has an additional advantage, that it is less likely to prematurely converge to a locally optimal solution. After all, the goal here is to discover novelty, not to converge to some optimized solution.

The major cost associated with this approach is to compute the novelty of individuals (or cocktails in this context). Novelty depends on how distinct an individual is by using a fitness function (the novelty metric of a cocktail) that favors more distinct individuals. The novelty of an individual is not only checked within its population, but also with all previous populations. 
%nin the complexity involved in calculating the fitness score, i.e., the novelty metric of a cocktail. The novelty metric calculation depends upon the computation of feature vectors of the cocktails. 
One possible way to tackle this problem is to observe whether the cocktails do reappear in subsequent generations or not. And if there is a significant number of overlaps, the feature vectors for those cocktails can be reused. This would help in decreasing the running time of the algorithm.

In summary, this thesis attempts to answer the following question:
\begin{itemize}
\item Can exploring a larger area of the chemical search space yield additional crystalline conditions?
\item Can the runtime of the algorithm be decreased by reducing the overhead associated with the fitness score calculation of the cocktails?
\end{itemize}

\section{Thesis Organization} \label{intro:organization}

This thesis is divided into five sections. %Section I gives a brief introduction of this research project. 
Chapter 2 describes the background in which we discuss some of the related works that have been done in protein crystallization screening. Chapter 3 focuses on the adaptation of the genetic algorithm while designing the fitness function. In Chapter 4, we discuss the experiments that were conducted in the lab and also compare the results of the algorithm described in this thesis with those of other algorithms. Chapter 5 concludes the thesis and proposes possible future enhancements.








 \chapter{Background} \label{ch:background}

\section{Related Work} \label{sec:bg-related}

\section{Summary} \label{rel:summary}
 \chapter{Method} \label{method}

% Data Acquisition
\section{Data Acquisition}
Training a classifier such as a SVM requires gathering as much data as possible because the algorithm generalizes better when it has been trained on a sufficiently large dataset. As per the definition of critical news, there are certain characteristics that a text should have in order to be labeled as a critical news. Due to this reason, a dataset does not exists that fits the categorization of text into critical and non-critical news. This prompted the creation of a new dataset for critical and non critical news.\par
The unlabelled dataset was obtained from the Tweets of the following Twitter users using the Twitter API\footnote{\url{https://developer.twitter.com/en/docs/tweets/timelines/api-reference/get-statuses-user_timeline.html}} and python library python-oauth2\footnote{\url{https://github.com/joestump/python-oauth2}}.

\begin{itemize}
    \item NYPD NEWS\footnote{\url{https://twitter.com/nypdnews}}
    \item Metropolitan Police\footnote{\url{https://twitter.com/metpoliceuk}}
    \item  Victoria Police\footnote{\url{https://twitter.com/VictoriaPolice}}
    \item Seattle Police Dept\footnote{\url{https://twitter.com/SeattlePD}}
    \item NYPDCounterterrorism\footnote{\url{https://twitter.com/NYPDCT}}
\end{itemize}

The data was stored locally in a sqlite\footnote{\url{https://www.sqlite.org/index.html}} database with the help of peewee\footnote{\url{http://docs.peewee-orm.com/en/latest/}} python library. A sample of data obtained from Twitter is as follows,

\begin{verbatim}
Watch: @PIX11News  gives an inside look at our #NeighborhoodPolicing
meetings on how they are connecting local NYPD police officers with
the community. https://t.co/D6K5DWxWWm https://t.co/NM0AWpPgja
\end{verbatim}

Each tweet was then manually labeled as either a critical news or non critical news and store in the database. The statistics of dataset is shown in \tableref{tbl:dataset_statistics}

\begin{table}
\begin{center}
\caption{Dataset statistics}
\label{tbl:dataset_statistics}
\begin{tabular}{@{}lccc@{}}
\toprule 
\rule[-1pt]{0pt}{14pt}Label&Number of samples\\
\midrule 
\rule[-1pt]{0pt}{14pt}Critical&1548\\
\rule[-1pt]{0pt}{14pt}Non-critical&595\\
\bottomrule
\end{tabular}
\end{center}
\end{table}

% Text Processing
\section{Text Processing}
\subsection{Tokenization}
Tokenization is the first step in processing a text which converts the text into a sequence of words. The tokenization is done with the help of NLTK\cite{loper2002nltk} python library which discards everything except words containing English alphabets. Furthermore, a regular expression was used to remove the URLs from the text.

\subsection{Stop word filtering}
There are several words such as "a", "the", "you" etc which are very common across every text. A significant improvement can be seen in most text classifiers by filtering out such words from the feature set\cite{silva2003importance}. These words that are generally filtered out during the text processing are known as stop words. The NLTK library contains lists of stop words that can be used to filter the tokens. For each token, a comparison is made with stop words one at a time and if a match is found the token is discarded from the feature set. The stop words for English text in NLTK library are as follows,

['i', 'me', 'my', 'myself', 'we', 'our', 'ours', 'ourselves', 'you', "you're", "you've", "you'll", "you'd", 'your', 'yours', 'yourself', 'yourselves', 'he', 'him', 'his', 'himself', 'she', "she's", 'her', 'hers', 'herself', 'it', "it's", 'its', 'itself', 'they', 'them', 'their', 'theirs', 'themselves', 'what', 'which', 'who', 'whom', 'this', 'that', "that'll", 'these', 'those', 'am', 'is', 'are', 'was', 'were', 'be', 'been', 'being', 'have', 'has', 'had', 'having', 'do', 'does', 'did', 'doing', 'a', 'an', 'the', 'and', 'but', 'if', 'or', 'because', 'as', 'until', 'while', 'of', 'at', 'by', 'for', 'with', 'about', 'against', 'between', 'into', 'through', 'during', 'before', 'after', 'above', 'below', 'to', 'from', 'up', 'down', 'in', 'out', 'on', 'off', 'over', 'under', 'again', 'further', 'then', 'once', 'here', 'there', 'when', 'where', 'why', 'how', 'all', 'any', 'both', 'each', 'few', 'more', 'most', 'other', 'some', 'such', 'no', 'nor', 'not', 'only', 'own', 'same', 'so', 'than', 'too', 'very', 's', 't', 'can', 'will', 'just', 'don', "don't", 'should', "should've", 'now', 'd', 'll', 'm', 'o', 're', 've', 'y', 'ain', 'aren', "aren't", 'couldn', "couldn't", 'didn', "didn't", 'doesn', "doesn't", 'hadn', "hadn't", 'hasn', "hasn't", 'haven', "haven't", 'isn', "isn't", 'ma', 'mightn', "mightn't", 'mustn', "mustn't", 'needn', "needn't", 'shan', "shan't", 'shouldn', "shouldn't", 'wasn', "wasn't", 'weren', "weren't", 'won', "won't", 'wouldn', "wouldn't"]

In addition to the above words, following words were also added to the stop word list in order to help the classifier generalize better.

['nypdprotect', 'nypdconnect', 'nypd', 'nypdoneil', 'nypdpsa', 'nypdchiefofdept', 'nypdbklynnorth', 'nypdtransit', 'nypddetect', 'nypdnew', 'nypdspecialop', 'nypdchiefpatrol', 'nypdhighway', 'nypdtransport', 'nypdconnect', 'nypdoneil', 'nycsafenypdorg', 'nypdct', 'sunday', 'monday', 'tuesday', 'wednesday', 'thursday', 'friday', 'saturday', 'seattlepd', 'seattledot', 'seattlefir', 'seattle', 'brooklyn', 'manhattan', 'one', 'two', 'three', 'four', 'five', 'six', 'vicpolicenew', 'victoriapolic', 'jointhemet', 'seattlefir', 'metpoliceuk', 'richmond', 'nycgov', 'https', 'nyc', 'nycdot']

\subsection{Stemming}
Stemming is the process of reducing the inflected words to their stem or root. For example, the word "happiness" can be reduced to it's stem "happy", the word "loved" can also be reduced to it's stem "love". By reducing the words to their stem, not only does the number of feature decrease the feature set itself improves as the inflected words of a same stem will be considered as a single feature instead of multiple features. The implementation of Porter's algorithm\cite{porter1980algorithm} found in NLTK was used for stemming the words.

\subsection{TF-IDF}
TF-IDF\cite{sparck1972statistical} is rather simpler feature weighing scheme that assigns weight to the words as per the occurrence in the text and it's importance in the corpus. TF-IDF for a term $t$ in a document $d$ is calculated as a product of two terms; Term Frequency and Inverse Document Frequency. 
\begin{equation}
    tfidf(t, d) = tf(t, d) * idf(t)
\end{equation}

The TF for a term in a document is the count of the term in the document. The IDF of a term is calculated as follows,
\begin{equation}
    idf(t) = \log{\frac{N}{1 + df(t)}}
\end{equation}

Where, $N$ is the number of total numer of documents and $df(t)$ is the number of documents that contains the term $t$.\par
The vectors thus obtained are then normalized using the $L^2$ norm.
\begin{equation}
    v_{norm} = \frac{v}{\lVert v \rVert}
\end{equation}

The scikit-learn\footnote{\url{http://scikit-learn.org/stable/index.html}} python library is used to obtain the TF-IDF values for the words.

% Text Processing
\section{Classification}
\subsection{Support Vector Machine}
Support Vector Machines non-linearly map the input vectors to a very high dimensional feature space and tries to construct a linear decision surface in the new feature space \cite{cortes1995support}. The SVM can use the kernel trick to change the structure of decision surface\cite{cortes1995support}. Some kernel that can be used with SVM are as follows,

Polynomial Kernel
\begin{equation}
    \label{eq:polynomial_kernel}
    K(x, x^{'}) = {(x.x^{'} + 1)}^{d}
\end{equation}

RBF Kernel
\begin{equation}
    \label{eq:rbf_kernel}
    K(x, x^{'}) = exp(- \frac{{\lVert x - x^{'} \rVert}^{2}}{2 \sigma^{2}})
\end{equation}
Where, $x$ and $x^{'}$ are the vector inputs, $d$ is the degree of the polynomial to use, and $\sigma$ is the smoothness parameter.

For the purpose of classifying critical and non-critical news, implementation of SVM with RBF kernel found in scikit-learn library was used.
 \chapter{Experiments and Results} \label{ch:results}
 \chapter{Conclusion}\label{conclusion}

\section{Future Works}
It has been empirically shown that most of the existing NLP system can be improved by using word embedding as extra word feature\cite{turian2010word}. However the extend of improvement that word embedding models could bring about for critical news detection remains to be   seen.\par
Another possibility that could be explored for improving the performance is using sequence analysis algorithms such as HMM\cite{baum1966statistical}, or Recurrent Neural Networks such as LSTM\cite{hochreiter1997long} and GRU\cite{cho2014learning} for class prediction. These algorithms could also be used along with the word embedding models.   

%   This set of LaTeX files uses the Appendix Package to handle
%   the appendices - it creates the "Appendices" page and puts
%   "Appendix A:" in the TOC.  So, strictly speaking, the
%   \appendix command is not necessary (because the appendix
%   package uses an "appendices environment").  But, for
%   the typedref package to operate correctly, you must include the
%   \appendix command - so, it's included after the beginning of
%   the appendices environment.  If you only have one appendix,
%   then you need to remove the "page" option from the uahdis.sty
%   file ('page' option occurs when loading appendix package).
%   You can add or subtract appendices - they are treated just like
%   chapters - the only difference is that they occur within the
%   appendices environment, which tells LaTeX to give them letters
%   instead of numbers.

%\begin{appendices}
%\appendix
%
%%   Include my appendices
%%\include{./CH7/appendix1name}
%%\include{./CH8/appendix2name}
%%\include{./CH9/appendix3name}
%
%%   Here comes the stuff after the main content
%%   turn off appendix environment/package first:
%\end{appendices}
\backmatter

%   Include my bibliography
\bibliographystyle{unsrt}
%\bibliographystyle{ieeetr}
\begin{singlespace}
\bibliography{references}
\end{singlespace}


%   Include my biography
%\include{./BACK/biography}
% MWT - biography not included in UAH thesis


%   We're done.
\end{document}
