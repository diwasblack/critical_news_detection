%   thesis.tex - UAH Master's Thesis Main TeX file
%
%   This is the main file for my thesis. It loads all content via
%   \include statements

%%% Making the DVI and PDF output the correct size %%%
%   Important - the default pagesize for DVI and PDF output in most LaTeX
%   distributions is a4 (210x297cm, or about 8.27×11.69 in), which won't work
%   for the UAH format.  So, you have to change the DVI and PDF output in your
%   LaTeX distribution.  In most distros, you need to change the following files:

%   (1.)  C:\texmf\dvipdfm\config\dvipdfmx.cfg
%   (2.)  C:\texmf\dvipdfm\config\config
%   (3.)  C:\texmf\pdftex\config\pdftex.cfg

%   For numbers (1.) and (2.), you will change where the files say "p a4" to
%   "p letter" (there should be some explanation in the files themselves, too).
%   For number (3.), you will change two lines of that file to read:

%   page_height 11 true in
%   page_width 8.5 true in

%   There are "model-thesislocal" versions of all the above files in this complete rar
%   package which 'should' override the a4 defaults - but these do not work in
%   some distributions, so editing the above files is usually necessary.
%   For more info, search google for "pdftex.cfg" and "dvipdfmx.cfg"
%%% (End DVI and PDF changes) %%%

%   Define the document class. The proper class is "book." The normal options
%   are "12pt" and "onesided". When printing out copies for purposes other than
%   submission to the Graduate School, the option "twoside" can be used to
%   format the output in a manner which looks good when printed duplex.

\documentclass[12pt,oneside,letterpaper]{book}
%\documentclass[12pt,twoside,letterpaper]{book}

%   Load the packages necessary for the thesis. The only required packages are
%   "uahdis", which loads the UAH Dissertation Style, and "chngpage" which is
%   used by "uahdis". The other packages are optional, and should be loaded only
%   if you have a need for the functionality they provide. Documentation on each is
%   available at CTAN.org. The basic purpose of each are described below:
%   units---provides a convenient mechanism for writing quantities with units.
%   bm---provides a convenient mechanism for producing bold symbols in math mode
%   hhline---improved lines and borders in tables
%   rotating---allows landscape oriented pages in a portrait oriented document
%   verbatim---prints text files verbatim. useful for computer code
%   amsfonts---provides additional typefaces
%   amssymb---provides additional mathematical symbols
%   xspace---properly adjusts space after \newcommands which expand to text
%   booktabs---creates traditional scientific tables - no vertical lines, very
%   few horizontal lines, with varying thickness (not meant to be used with hline)

%   Excellent descriptions of all LaTeX packages can be found here:
%   http://www.tug.org/tex-archive/help/Catalogue/index.html

\usepackage[]{uahdis,chngpage,units,bm,hhline,rotating,verbatim,amsfonts,amssymb,xspace}
\usepackage[]{indentfirst,layouts, booktabs, enumerate, gensymb}
\usepackage{color,soul}

%My Packages
\usepackage[hyphens]{url}
\usepackage[none]{hyphenat}
\usepackage{amsmath}
% \usepackage{hyperref}
\usepackage{array, graphicx}
\usepackage{multirow}
\usepackage{microtype}
\usepackage{float}
\usepackage{placeins}
\usepackage{csquotes}
\usepackage{longtable}

%\usepackage{url} % hyperref works too
%\urlstyle{same}  % (sf also works, for something more subtle than tt)

%   Load the tex file which contains all local \newcommands and \newenvironments

%%%
%%% Environments
%%%

\newsavebox{\speaker}
\newenvironment{chapterquote}[1]
    {\begin{flushright}\begin{minipage}{3 in}\sbox{\speaker}{#1}\itshape\singlespace}
    {\begin{flushright}---\usebox{\speaker}\end{flushright}\end{minipage}\end{flushright}}

% Note: chapterquote works (and looks) best when the chapter begins with a \section{} ...
% If you hadn't planned on beginning the chapter with a \section, try \section{Overview} :-)
% otherwise, some \vspace{} might be necessary, but that won't be consistent throughout the thesis

%%%
%%%  Functions---Commands that take parameters
%%%
\newcommand{\cfig}[3]{\centering\includegraphics[keepaspectratio=true,width=#3in]{./CH#1/EPSFDocs/#2}}
\newcommand{\acro}[1]{\textsc{#1}}
\newcommand{\sci}[2]{\ensuremath{#1 \!  \times \!  10^{#2}}}
\newcommand{\vect}[1]{\boldsymbol{\mathbf{#1}}}
\newcommand{\unitvec}[1]{\vect{\hat{#1}}}
\newcommand{\threebythree}[9]{\renewcommand{\arraystretch}{0.75}\begin{vmatrix}#1&#2&#3\\#4&#5&#6\\#7&#8&#9\end{vmatrix}\renewcommand{\arraystretch}{1.0}}
\newcommand{\mysci}[2]{\ensuremath{#1 \!  \times \!  10^{#2}}}
\newcommand{\myprop}[5]{g(#1,#2;#3,#4;#5)}
\newcommand{\myint}[3]{\int_{#1}^{#2}#3}
\newcommand{\myintinfinf}[1]{\myint{-\infty}{\infty}{#1}}
\newcommand{\minus}{\scalebox{0.75}[1.0]{$-$}}


%%%
%%%  New Math Operators
%%%

\DeclareMathOperator{\polylog}{Li}


%%%
%%%  Symbols---Commands that are shorthand
%%%
\newcommand{\rf}{\acro{rf}\xspace}
\newcommand{\dc}{\acro{dc}\xspace}
\newcommand{\ac}{\acro{ac}\xspace}
\newcommand{\ccd}{\acro{ccd}\xspace}
\newcommand{\mach}{\acro{mach2}\xspace}
\newcommand{\mgmhd}{\acro{mgmhd}\xspace}
\newcommand{\cea}{\acro{cea}\xspace}
\newcommand{\half}{\nicefrac{1}{2}\,}
\newcommand{\xx}{\ensuremath{\unitvec{x}}\xspace}
\newcommand{\yy}{\ensuremath{\unitvec{y}}\xspace}
\newcommand{\zz}{\ensuremath{\unitvec{z}}\xspace}
\newcommand{\ttl}{\acro{ttl}\xspace}
\newcommand{\pmt}{\acro{pmt}\xspace}
\newcommand{\eg}{\textit{e.g.}\xspace}
\newcommand{\ie}{\textit{i.e.}\xspace}
\newcommand{\qmn}{\ensuremath{q^{mn}}\xspace}
\newcommand{\isp}{\ensuremath{I_{sp}}\xspace}
\newcommand{\wrt}{\textit{w.r.t.}\xspace}


%%%
%%%  Other stuff I've found - Evaluate for usefulness
%%%

%\newcommand{\quan}[2][]{\mbox{$#1\,\mathrm{#2}$}}
%\newcommand{\vv}[1]{\ensuremath{\boldsymbol{#1}}}
%\newcommand{\tempc}[1]{\quan[#1]{^{\circ}C}}
%\DeclareMathOperator{\sinc}{sinc}
%\newcommand{\comment}[1]{\marginpar{\Large \hfill \ddag}\textsf{#1}}
%\newcommand{\comment}[1]{}


%%% my definition added 

%   The next line requires the leading "%". It is only useful if you are using
%   WinEdt as your text editor. It allows WinEdt to collect bibliographic entries into
%   a pop-up table that you can summon when you are \cite-ing a source.

%GATHER{Bibliography.bib}

%   Define my name
\author{Diwas Sharma}

%   Define the title of my thesis (user upper and lower case)
\title{Significant News Detection}

%   Define the year
\date{2018}

%   Define my department
\uahdepartment{Computer Science}

%   Define my advisor (no title - i.e., no "Dr.")
\uahadvisor{Ramazan Aygun}

%   Define my committee members (no titles)
\uahmema{Huaming Zhang}

\uahmemb{Vineetha Menon}

% MWT - UAH Thesis - only three committee members
% if more than three, uncomment lines below
% and appropriate lines in uahdis.sty

%\uahmemc{Fourth Committee Member}

%\uahmemd{Fifth Committee Member}

%MWT

%   Define my Department Chair (no title)
\uahdeptchair{Heggere S. Ranganath}

%   Define my College (do not write "College of")
\uahcollege{Science}

%   Define my College Dean (no title)
%   the tex for Dean Aunon is:  Jorge I. Au\~n\a'on
\uahcolldean{Sundar Christopher}

%   Define my Degree (i.e., Master of Science, Doctor of Philosophy)
\uahdegree{Master of Science in Computer Science}

%   Define my Program name
\uahprogram{Computer Science}

%   Shortened name for degree
%   type "master's" or "doctoral" (without quotes)
\uahdegreeshort{master's}

%   Define my Document Type
\uahdoctype{thesis}

%   Define my Graduate Dean
\uahgraddean{David Berkowitz}


%   Let's get started (Finally!)
\begin{document}

%   To view a layout of the margins of this document,
%   uncomment the below line (requires "layouts" package):
%\layout
%\tocdiagram\tocdesign

%   Here comes the stuff that goes before the main content
\frontmatter

\pagestyle{plain}

\maketitle

\copyrightpage

%MWT - UAH Approval Form
\approvalpage

%\makeabstract % Now issued in ./FRONT/abstract.tex

%   Include my abstract
\chapter*{Abstract}
% the \makeabstract command creates the top portion of the abstract
% page ... must be issued before the abstract content
\makeabstract

%%%%%%%%%%% Your Abstract Text Goes after Here %%%%%%%%%%%%%%%%%%%%%%%
Social media platforms nowadays have a large number of fake or false news which have been misleading and negatively impacting viewers. In order to combat the problem, being able to differentiate important news stories which need to be verified from unimportant news stories which need not, would be a decent starting point. In this thesis, we introduce "significant news" and define it as news that affects a large number of people, changes the routines of daily life, and needs verification on the information presented. This thesis then explores if it is possible to construct a classifier for detecting the significant news articles.

A dataset containing 1548 significant and 595 non-significant articles was prepared by manually labelling the posts obtained from Twitter. Various classifiers including logistic regression, support vector machine, random forest, and neural network -- were trained on the dataset. They each achieved an accuracy greater than 90 percent, with the neural network model achieving the highest accuracy of 93.654 percent. This indicates that it is in fact possible to build fairly accurate classifiers for detecting significant news. This thesis then describes a few possible future directions that could be explored for further improving the performance of significant news detection.

%%%%%%%%%%%%% Your Abstract Text should be before Here %%%%%%%%%%%%%%

% the abstractsig command creates the signature spaces after the
% abstract, and therefore, must be issued after the abstract.
\abstractsig


% Abstract signatures:
%\abstractsig % command now issued in abstract.tex

%   Include my acknowledgements
\chapter*{Acknowledgments}

I would like to express my very great appreciation to Dr. Ramazan Aygun for his guidance and support throughout this research.

Also, I would also like extend my thanks to my family and friends who supported me in my endeavors.


%   Make a Table of Contents
\tableofcontents

%   Make a List of Figures
\listoffigures

%   Make a List of Tables
\listoftables

%   Make a List of Symbols (Comment out if unwanted)
\listofsymbols

\noindent Acronyms

\symboldefinition{API}{Application Program Interface}
\symboldefinition{CNN}{Convolution Neural Network}
\symboldefinition{IDF}{Inverse Document Frequency}
\symboldefinition{LSTM}{Long Short Term Memory}
\symboldefinition{NLP}{Natural Language Processing}
\symboldefinition{RBF}{Radial Basis Function}
\symboldefinition{SVM}{Support Vector Machine}
\symboldefinition{TF}{Term Frequency}
\symboldefinition{t-SNE}{T-distributed Stochastic Neighbor Embedding}
\symboldefinition{SGD}{Stochastic Gradient Descent}
\symboldefinition{VSM}{Vector Space Modeling}

%   Make a List of Abbreviations (Comment out if unwanted)
% \include{./FRONT/abbreviations}

%   Make a "Chapter" header in TOC, per UAH style
%   issue command after last frontmatter TOC entry
%   cannot come directly before first chapter \include
\addchapheadtotoc

%   Set my dedication (optional - comment out if unwanted)
%\dedication{}

%   Make Epigraph Page (Optional - comment out if unwanted)

\clearpage

\pagestyle{myheadings} \markright{}

%   Here comes the main content
\mainmatter

%   Include my chapters
%   LaTeX will look for your chapter files in the appropriate folders,
%   as addressed below.  LaTeX will look for a .tex file with the
%   same name as the name you give it below - i.e., for the CH1 folder,
%   LaTeX will look for introduction.tex (which is where you will type
%   all your chapter 1 stuff).  You can name your chapters/files whatever
%   you want - just make sure the names below match the names in the
%   folders.  Also, you can add or subtract chapters as you like -
%   just make sure that the address and filenames below match your
%   file structure.

 \chapter{Introduction}\label{introduction}

% Motivation
\section{Motivation}\label{intro:motivation}
Based on a recent study, 68 percent of US adults said they at least occasionally get news on social media; however, 57 percent of those people expect the news to be largely inaccurate \cite{matsa2018news}. These facts portray a significant problem with the social media platforms: verifying the truthfulness of the articles posted on them. Recently, the term ``fake news" has soared in popularity across various social media, news industry, and research papers alike. Jaster et al. define the term as \cite{gelfert2018fake}

\enquote{Fake news is the deliberate presentation of (typically) false or misleading claims as news, where the claims are misleading by design.}

\noindent
A study has shown that fake news viewing impacts political attitudes, enhances the feelings of inefficacy, alienation, and cynicism toward politicians \cite{balmas2014fake}. And, others have pointed out that fake news could impact the ability of people to accept truthful news by confusing them with false stories \footnote{\url{https://www.nytimes.com/2016/11/28/opinion/fake-news-and-the-internet-shell-game.html?\%20r=0}}.

During the United States presidential election of 2016, fake news propagated far more rapidly on Facebook than did real news from mainstream media\cite{silverman2016analysis}, and it was also reported that 75 percent of people who saw fake news stories at that time said that they believed them\cite{silverman2016most}. There were fake news for both candidates despite one candidate had more fake news in favor than the other candidate \cite{silverman2016analysis}. This led to a number of commentators proposing that the fake news influenced the election  \cite{parkinson2016click, read2016donald, dewey2016facebook}

To control the effects that fake news might have, it is essential to develop systems that can identify fake news as early as possible. However, automatic detection of fake news poses several problems. Firstly, in order to achieve its purpose of misleading readers, a fake news story will try its best to appear genuine. Therefore, based solely on content, it is difficult to classify the story as fake. Secondly, using auxiliary information such as knowledge base and social engagements actually leads to another problem of verifying the quality of the data itself\cite{shu2017fake}. It is also impractical for the reviewers to analyze every single article, statement or message that is found in social media and decide whether it is a fake news. Due to these reasons, a number of researchers have even proposed that a better compromise could be a semi-automated system where an algorithm would augment a human reviewer to identify fake news \cite{conroy2015automatic, chen2015news, wiegand2016veracity}.

One key observation that can be fundamental in combating fake news is that we do not need to analyze every single article, message or statement. There are articles that describe topics such as infotainment, personal news, beauty tips, etc. which might not have severe impact even if they were fake. Such articles can be excluded from the verification step or can be verified as needed. For example, consider the following statement: ``I watched the Black Panther movie in the weekend. Fantastic effects!". Detecting whether the person may have watched the movie or not, or whether the movie has fantastic effects may not be considered as important. However, a statement such as ``There has been a shooting at the mall" needs to be verified.
To summarize, we could say that there are somewhat important articles that needs to be verified and there are others which may be excluded from the verification step.

% Research Problem
\section{Research Problem} \label{intro:research}
The concept of separating news into somewhat important news which needs to be verified and other slightly unimportant news could be in a way compared to the dichotomy of news into hard and soft. There are several definitions for hard and soft news \cite{reinemann2012hard, shoemaker2012news} but the one by Reinemann et al., 2012 is simple: 

\enquote{Hard news is defined as reports about politics, public administration, the economy, science, technology and related topics. Soft news is defined as reports about celebrities, human interest, sport and other entertainment-centred stories.}

The categorization of text into hard news and soft news is inadequate for the purpose of identifying important article from unimportant ones because there might be articles in hard news which are not that important and soft news might also have articles which are important. So a new labelling "significant" is introduced which is defined as follows:

\enquote{A text is labeled as significant if it affects a large number of people, changes the routines of daily life, and needs verification on the information presented.}

Any article whose content satisfies the conditions shown in \tableref{tbl:significant_preconditions} could be labelled as significant and all other articles which does not meet those should be labelled as a non-significant.

\begin{table}[h]
\begin{center}
\caption{Preconditions for a text to be significant}
\label{tbl:significant_preconditions}
\begin{tabular}{l | r}
\toprule
SN & Condition \\
\midrule
1 & Content affects large number of people. \\
2 & Content changes their routines of daily life. \\
3 & Content needs to be verified. \\
\bottomrule
\end{tabular}
\end{center}
\end{table}

% Any article whose content affects large number of people, changes their routines of daily life, and needs to be verified could be labelled as significant and all other articles which does not meet those criteria should be labelled as a non-significant. 

Many research studies aim at detecting fake news. Although there already exist datasets for studying fake news, there is no dataset available for detecting significant news as mentioned in this thesis. To study this problem, we work on the following problems:

\begin{enumerate}
    \item Building a new dataset for significant news detection.
    \item Studying existing fake news detection methods for detecting significant news.
    \item Selecting features from fake news detection methods and evaluating a number of classifier models for significant news detection.
\end{enumerate}

Especially, building the dataset for significant news detection was complex. Finding ground-truth or reference messages for labeling was important for this study. Once these messages are labeled based on the definition of significant news, we built a number of classifiers. Therefore, the eventual goal is to be able to construct a binary classifier that is able to separate significant news from non-significant news based on its content. Once these news statements are detected as significant, they can be further evaluated to determine whether they are fake or not.

% Thesis Organization
\section{Thesis Organization} \label{intro:organization}
The remainder of the thesis is organized as follows. Chapter 2 covers brief background for the text classification problems and provides research studies that are related to this thesis. Chapter 3 presents how the dataset is built and the methodology that was used to train, validate and test various classifier models. Chapter 4  provides the results and analysis of the experiments. And finally, Chapter 5 concludes the thesis with possible future directions that could be explored.
 \chapter{Background} \label{background}
The goal of text classification is to automatically assign a document to one of the pre-defined classes using the information available from the content and the meta-data of the document.

For most text processing systems, the stream of character representation of the text needs to be converted to another suitable representation that can be used with the analysis algorithms. A popular way of representing a text is to convert it into vectors that somehow capture the essence of the document. 

The vectors can then be used to construct predictive models that can assign class labels to the documents. The complexity of such models could range from simple hand-crafted rule-based classifiers to models that can learn from a labelled dataset. Furthermore, the label assignment itself could be categorized as hard or soft based on whether the assignment is explicit -- assigning the label itself to the text -- or implicit -- assigning a probability value. 

\section{Feature Generation}
\subsection{Tokenization}
Tokenization is the first step in text processing which converts the stream of characters representation of the text to a sequence of words by removing the white space characters and punctuation symbols between the words. An example of tokenization on a given text is shown in \tableref{tbl:tokenization_example}.

\begin{table}[h]
\begin{center}
\caption{Example of tokenization}
\label{tbl:tokenization_example}
\begin{tabular}{p{6cm}p{6cm}}
\toprule 
Text&Tokens\\
\midrule 
"Don't go there."&["Don't", "go", "there"]\\
\bottomrule
\end{tabular}
\end{center}
\end{table}

\subsection{Stop word filtering}
Words such as "a", "the", "you", etc. are very common across English text documents, so the occurrence of these words is a poor indicator of whether the text belongs to a particular class. A common practice is to filter out these words for the feature set, and it has been empirically shown that filtering such words improves the performance on most text classification tasks \cite{silva2003importance}. These words that are filtered out during the processing are known as stop words. 

\subsection{Stemming}
Stemming is the process of reducing the inflected words to their stem or root so that they represent the same feature during the processing. For example, the word "happiness" can be reduced to its stem "happy", and the word "loved" can also be reduced to its stem "love". By reducing the words to their stems, not only decreases the number of features, but also improves the feature set itself  as the inflected words of the same stem will be considered as a single feature instead of multiple features. 

\subsection{Vector Space Model}
The vector space model (VSM) or bag-of-words model represents a text as a collection of words without regards to the sequence in which they occur. In VSM, each unique word in the vocabulary is treated as a feature of the text because of which it usually ends up with a large number of features. For such cases, the feature extraction algorithms such as information gain, gini index, latent semantic indexing, linearized singular value decomposition, etc. can be used for selecting the relevant features from the original feature set \cite{aggarwal2012survey}.

\subsection{n-gram models}
The n-gram model of a text is a probabilistic model that represents text as a continuous sequence of $n$ words in a such way that the context of the word is captured along with the word itself. The VSM is a special case of the n-gram model where the value of $n$ is 1. Some of the n-gram representation of the text "The quick brown fox jumps." is shown in \tableref{tbl:ngram_representation}.

\begin{table}[h]
\begin{center}
\caption{Example ngram representation}
\label{tbl:ngram_representation}
\begin{tabular}{p{3cm}p{8cm}}
\toprule 
Model&Representation\\
\midrule 
Unigram($n=1$)&"The", "quick", "brown", "fox", "jumps" \\
Bigram($n=2$)&"The quick", "quick brown", "brown fox", "fox jumps" \\
Trigram($n=3$)& "The quick brown", "quick brown fox", "brown fox jumps" \\
\bottomrule
\end{tabular}
\end{center}
\end{table}

\subsection{TF-IDF}
TF-IDF\cite{sparck1972statistical} is a rather simple feature weighing scheme that assigns some weight to a term based on its occurrence in the text and its importance in the corpus. TF-IDF for a term $t$ in a document $d$ is calculated as a product of two terms: Term Frequency ($tf$) and Inverse Document Frequency ($idf$). 
\begin{equation}
    \label{eq:tf_idf_equation}
    tfidf(t, d) = tf(t, d) * idf(t)
\end{equation}

The $tf$ for a term in a document is the count of the term in the document and $idf$ for a term is calculated as follows:
\begin{equation}
    \label{eq:idf_equation}
    idf(t) = \log{\frac{N}{1 + df(t)}}
\end{equation}

\noindent
where $N$ is the number of total number of documents and $df(t)$ is the number of documents that contains the term $t$.

\subsection{Word Embedding Models}
Lately, there has been a significant interest and development in word embedding models \cite{mikolov2013distributed, pennington2014glove, bojanowski2017enriching}. In a word embedding model, each word is represented by a vector in a feature space such that similar words will be represented by similar vectors\cite{mikolov2013distributed}. The word vectors are then combined to create a vector representation of the text which by definition will force two texts that are similar in content but different in words to have a similar vector representation.

% Related Work
\section{Related Work on Classifying Text} \label{related_works}
Due to the way the significant news is defined in this thesis, there is not much research work that could be used for direct comparison. The closest classification that could be compared is the dichotomy of hard/soft news but the hard/soft classification of news is mostly confined to journalism and an actual implementation of such classifiers is rarely found. The majority of research instead focuses on the much broader problem of topic classification for various purposes \cite{wang2012baselines, lee2011twitter, joachims1998text, nigam2000text}.

The topic classification algorithm could be used for hard/soft news classification by using the labelling technique shown in \tableref{tbl:topics_to_hard_soft}. However, the soundness of such a method for hard/soft news classification remains unknown for now.

\begin{table}[h]
\begin{center}
\caption{Conversion of topic labels to hard/soft news labels}
\label{tbl:topics_to_hard_soft}
\begin{tabular}{lr}
\toprule 
Topics&Label \\
\midrule 
politics, public administration, science, technology, etc & Hard\\
\hline
celebrities, human interest, sport, entertainment, etc & Soft\\
\bottomrule
\end{tabular}
\end{center}
\end{table}

Lately, the fake news detection problem has received a lot of interest from researchers. A fake news detection algorithm could use any combination of information among news content, social content (e.g. user's engagement), and dynamic information (e.g. news propagation) in order to perform the prediction \cite{shu2018fakenewsnet}; however, even with a multitude of such information, it is not easy to construct an automated algorithm for fake news detection\cite{shu2017fake}.

One of the simplest models that was successfully implemented for fake news detection is the naive bayes classifier which was trained on a modified BuzzFeedNews\footnote{\url{https://github.com/BuzzFeedNews/2016-10-facebook-fact-check}} dataset containing Facebook posts of news articles \cite{granik2017fake}. The original dataset consists of four labels for truthfulness rating which are distributed as shown in \tableref{tbl:buzzfeed_dataset}. In the research, the number of classes were reduced to two by discarding articles which were labelled as "mixture of true and false" or "no factual content" along with articles without any text in them. The classifier was then trained using the remaining 1771 news articles for the binary classification problem. The model was then able to achieve a test accuracy of 74 percent which is a good result given the simplicity of the model.

\begin{table}[h]
\begin{center}
\caption{Statistics about BuzzFeed dataset}
\label{tbl:buzzfeed_dataset}
\begin{tabular}{lr}
\toprule 
Label&Number of samples \\
\midrule 
mostly false&104 \\
mixture of true and false&245 \\
mostly true&1669 \\
no factual content&264 \\
\midrule
&2282 \\
\bottomrule
\end{tabular}
\end{center}
\end{table}

The research by Gilda, 2017 provides the most exhaustive performance comparison of standard classification algorithms for the fake news detection problem \cite{gilda2017evaluating}. The research contrasts the performance of various classification algorithms  (support vector machines (SVM), stochastic gradient descent (SGD), random forest, bounded decision trees, and gradient boosting) by training them on a modified Signal Media news dataset \cite{corney2016million} which has labeled articles from verified reliable sources as truthful and articles from verified unreliable sources as fake. The distribution of the labelled dataset that was obtained from the original unlabelled 1 million articles is shown in \tableref{tbl:reduced_signal_media_dataset}. The performance comparison of the standard classifiers on the new dataset when using a TF-IDF as the feature set is shown in \tableref{tbl:signal_media_performance} \cite{gilda2017evaluating}. It empirically shows that most of the standard classifiers work reasonably well for fake news detection on the Signal Media dataset.

\begin{table}[h]
\begin{center}
\caption{Reduced Signal Media dataset for Fake News Detection}
\label{tbl:reduced_signal_media_dataset}
\begin{tabular}{lr}
\toprule 
Label&Number of samples \\
\midrule 
fake&3217 \\
truthful&7834 \\
\midrule
&11051 \\
\bottomrule
\end{tabular}
\end{center}
\end{table}

\begin{table}[h]
\begin{center}
\caption{Performance of models in reduced Signal Media dataset}
\label{tbl:signal_media_performance}
\begin{tabular}{lr}
\toprule 
Model&Test Accuracy\\
\midrule 
Naive Bayes&67.89\\
Bounded Decision Trees&66.1\\
Gradient Boosting&68.7\\
Random Forests&67.6\\
Stochastic Gradient Descent&77.2\\
Support Vector Machine&76.2\\
\bottomrule
\end{tabular}
\end{center}
\end{table}

However, for the LIAR\cite{wang2017liar} dataset the performance of the standard classifiers are a bit underwhelming as shown in \tableref{tbl:liar_performance} \cite{wang2017liar}.  Note that this is a multi-class classification rather than a binary classification due to the number of labels in the dataset. Nevertheless, the accuracy is not impressive and this could be due to the dataset.  The LIAR dataset is human-labeled short statements obtained from POLITIFACT\footnote{\url{https://www.politifact.com/}} API which contains six labels assigned based on their truthfulness rating: "pants-fire", "false", "barely-true", "half-true", "mostly-true", and "true". The distribution of the labels in the dataset is mostly balanced except for 1050 "pants-fire" cases while for all other labels it ranges from 2063 to 2638. Since the articles are short and require fact-checking to determine their truthfulness, it is much harder for the classifier to perform the prediction.

\begin{table}[h]
\begin{center}
\caption{Performance of models in LIAR dataset}
\label{tbl:liar_performance}
\begin{tabular}{lccc}
\toprule 
Model&Test Accuracy\\
\midrule 
Logistic Regression&24.7\\
SVM&25.5\\
Bidirectional LSTM&23.3\\
CNN&27.0\\
\bottomrule
\end{tabular}
\end{center}
\end{table}

Unsatisfactory performance of detecting fake news solely from news content has led researchers to investigate the relation between users on social media and news \cite{shu2018understanding}. The research suggested that the features of users who were more likely to trust fake news over real news were different from those of users who were more likely to trust real news. The TCNN-URG model for fake news detection by Qian et. al., 2018 was able to use this information to its advantage. The model combined the information from news content with dynamic information (historical responses by the user) of the article to perform the prediction. The performance of the model was then compared with standard classification techniques by training them on the Weibo dataset\cite{ma2016detecting} which consisted of 2313 rumors and 2351 non-rumor articles obtained using the API of Sina Weibo. The performance achieved by the models is shown in \tableref{tbl:weibo_performance} \cite{qian2018neural}. It was observed that TCNN-URG model performed well compared to the previous models that only utilized the information from news content.

\begin{table}[h]
\begin{center}
\caption{Performance of models in Weibo dataset}
\label{tbl:weibo_performance}
\begin{tabular}{lccc}
\toprule 
\rule[-1pt]{0pt}{14pt}Model&Test Accuracy\\
\midrule 
SVM with LIWC&66.06\\
SVM with POS-gram&74.77\\
SVM with 1-gram&84.76\\
CNN&86.23\\
TCNN&88.08\\
TCNN-URG&89.84\\
\bottomrule
\end{tabular}
\end{center}
\end{table}

One study that tries to tackle fake news detection from an entirely different perspective is the research by Chen et. al, 2015. The research starts with the fact that news with clickbait headlines often had content which were misleading and unverified \cite{silverman2016analysis}. Using that premise with the fact that news with clickbait headlines had been the major factor responsible for the spread of fake news \cite{silverman2015lies}. The research proposed a clickbait article detection system as a means of identifying the fake news \cite{chen2015misleading}.

\section{Summary}
This chapter provides a brief explanation of text classification and explains the concepts related to text processing: tokenization, stop words filtering, stemming, vector space model, n-gram model, TF-IDF, and word embedding models. The chapter then describes various related research done in the field of fake news detection.
 \chapter{Method} \label{method}
The first step of constructing a classifier for critical news is to acquire a dataset containing the labels for critical and non-critical news. Once the dataset is obtained, the texts are then processed to obtain the unigram terms. The IDF values for the terms are then calculated by using the equation \eqref{eq:idf_equation}, which are then used to compute the TF-IDF vectors for the text.

The vectors for documents are then randomly divided into a training set and a test set. The training set is used to train the models and optimize the hyperparamters by using a cross validation technique. And, the test set is used to evaluate the performance of the best performing models from the cross validation step. The steps that were followed in order to train, validate, and test the classifier models are shown in \figureref{fig:training}.

\begin{figure}[h]
    \cfig{3}{training.png}{3}
    \caption{Flow diagram for training and testing}
    \label{fig:training}
\end{figure}

Once the models are trained, they can be used to predict the class assignment for new articles. The first step would be convert the piece of text to a vector by using the TF-IDF algorithm. Then, the vector can be used with a trained classifier model to predict the class label. The steps that should used in order to perform the classification is summarized in \figureref{fig:prediction}.

\begin{figure}[h]
    \cfig{3}{prediction.png}{3}
    \caption{Flow diagram for predicting class label using pre-trained model}
    \label{fig:prediction}
\end{figure}

\section{Dataset}
As per the definition, there are certain characteristics that a piece of text should satisfy in order to be labeled as critical news. Due to this reason, a dataset that matches categorization of text into critical and non-critical news does not exist. This prompted the creation of a new dataset for critical news detection.

\subsection{Data Acquisition}
The unlabelled dataset was obtained from the posts of Twitter users shown in \tableref{tbl:twitter_users}. The posts were obtained with the help of python-oauth2\footnote{\url{https://github.com/joestump/python-oauth2}} library using the official Twitter API\footnote{\url{https://developer.twitter.com/en/docs/tweets/timelines/api-reference/get-statuses-user_timeline.html}}.

\begin{table}
    \centering
    \caption{Twitter accounts used for collecting data}
    \label{tbl:twitter_users}
    \begin{tabular}{c|c}
    \toprule
    Account & Link \\
    \midrule
    NYPD NEWS & \url{https://twitter.com/nypdnews} \\
    Metropolitan Police & \url{https://twitter.com/metpoliceuk} \\
    Victoria Police & \url{https://twitter.com/VictoriaPolice} \\
    Seattle Police Dept & \url{https://twitter.com/SeattlePD} \\
    NYPDCounterterrorism & \url{https://twitter.com/NYPDCT} \\
    \bottomrule
    \end{tabular}
\end{table}

The data was then stored locally in an sqlite\footnote{\url{https://www.sqlite.org/index.html}} database with the help of peewee\footnote{\url{http://docs.peewee-orm.com/en/latest/}} python library. A sample of data obtained from Twitter is as follows,

\begin{verbatim}
Watch: @PIX11News  gives an inside look at our #NeighborhoodPolicing
meetings on how they are connecting local NYPD police officers with
the community. https://t.co/D6K5DWxWWm https://t.co/NM0AWpPgja
\end{verbatim}

\subsection{Labelling}
Each tweet was then manually labeled as either critical news or non-critical news based on the pre-conditions. The statistics of the new dataset constructed is shown in  \tableref{tbl:dataset_statistics} and some examples of the labels and their explanation are given below.

\begin{table}[h]
\begin{center}
\caption{Critical news dataset statistics}
\label{tbl:dataset_statistics}
\begin{tabular}{lr}
\toprule 
Label&Number of samples\\
\midrule 
Critical&1548\\
Non-critical&595\\
\bottomrule
\end{tabular}
\end{center}
\end{table}

\subsubsection{Labelling for Sample 1}
\textbf{Text:}
\textit{Great news!! Missing 20-year-old Melissa Wilson has been located safe and well. Thank you for the shares!}\par
\textbf{Label:} Critical\par
\textbf{Explanation:} A number of people might still be looking for the missing person either actively or inactively. Any updates to the whereabouts of the person would be an important piece of information for those people. The text mentions that the missing person has been found and is safe and well. If the missing person has in fact been located, the people searching for the person could stop searching for the person and can continue with their other tasks. Since the content present affects a large number of people, changes their routines of their daily life, and needs to be verified, it would be labelled as critical.

\subsubsection{Labelling for Sample 2}
\textbf{Text:}
\textit{Detectives investigating the murder of Kwabena Nelson in Tottenham have made an arrest Haringey.}\par
\textbf{Label:} Critical\par
\textbf{Explanation:} A murder of a person induces fear in a number of people who might be living in the same location where the crime is supposed to have occurred. The people living in the area might avoid walking alone in the neighbourhood at night, cancel their plans of moving in an apartment near the area or even move out of the area after knowing about it. Knowing that the detectives made an arrest on the crime would be a very important piece information that might bring them some peace of mind. As the content affect a large number of people, changes the routines of their daily life and needs to be verified, it would be labelled as critical.

\subsubsection{Labelling for Sample 3}
\textbf{Text:}
\textit{Great example of NYPDconnecting in the Bronx. NYPD49Pct Neighbor- hoodPolicing officers worked with the community to address a garbage condition on a resident block in their neighborhood.}\par
\textbf{Label:} Non-Critical\par
\textbf{Explanation:} The number of people the text could concern would be physically limited to the people living around that neighborhood. And, even for those people, addressing a garbage condition would be a minor concern. The information that the problem was resolved could hardly change their behavior. For most of them, their life would continue the same way even after obtaining the information. As the number of people that the text affects is small and could hardly change their daily routine, the text should be labelled as non-critical.

\subsubsection{Labelling for Sample 4}
\textbf{Text:}
\textit{It was over before it began for this 18-year-old who lost her licence after only two hours.}\par
\textbf{Label:} Non-Critical\par
\textbf{Explanation:} The information in the content is of a personal nature which most of the time is essential only for immediate family members and friends. And, even for those people, the information that the person lost their licence could hardly be a matter of great importance as long as the person is safe. Since the content does not describe any unfortunate events, the only new information would be that the person lost their licence. For most of the immediate family and friends their life would continue the same way as before even after reading about the news. Since it fails to affect the daily routines of a significant number of people it should be labelled as non-critical.

% Text Processing
\section{Feature Generation}
\subsection{Tokenization}
The tokenization is done with the help of NLTK\cite{loper2002nltk} python library which discards everything except English alphabets.  Tokenization done in this way will also discard the numeric characters such as "1", "2", "3", etc from the text due to which the articles could loses the temporal information in the content. The tokenizer first removes all of the punctuation symbols from a piece of text and then separates the words by using the white space characters. Furthermore, a regular expression was used to remove the URLs from the text. An example of the output generated by the tokenizer is shown in \tableref{tbl:tokenization_sample}

\begin{table}[h]
\begin{center}
\caption{Sample of tokenization}
\label{tbl:tokenization_sample}
\begin{tabular}{p{8cm}p{5cm}}
\toprule 
Text&Tokens\\
\midrule 
Watch: @PIX11News  gives an inside look at our \#NeighborhoodPolicing
meetings on how they are connecting local NYPD police officers with
the community. \url{https://t.co/D6K5DWxWWm} \url{https://t.co/NM0AWpPgja} & "watch", "pix", "news", "gives", "an", "inside", "look", "at", "our", "neighborhoodpolicing", "meetings", "on", "how", "they", "are", "connecting", "local", "nypd", "police", "officers", "with", "the", "community" \\
\bottomrule
\end{tabular}
\end{center}
\end{table}

\subsection{Stop word filtering}
The NLTK library contains lists of stop words for English text that can be used to filter the tokens. For each token, a comparison is made with stop words one at a time and if a match is found the token is discarded from the feature set. \tableref{tbl:stop_words} shows the list of stop word present in the NLTK library.

\begin{table}
\begin{center}
\caption{English stopwords in nltk}
\label{tbl:stop_words}
\begin{tabular}{lcccccr}
\toprule
i&me&my&myself&we&our&ours\\ \hline 
ourselves&you&you're&you've&you'll&you'd&your\\ \hline 
yours&yourself&yourselves&he&him&his&himself\\ \hline 
she&she's&her&hers&herself&it&it's\\ \hline 
its&itself&they&them&their&theirs&themselves\\ \hline 
what&which&who&whom&this&that&that'll\\ \hline 
these&those&am&is&are&was&were\\ \hline 
be&been&being&have&has&had&having\\ \hline 
do&does&did&doing&a&an&the\\ \hline 
and&but&if&or&because&as&until\\ \hline 
while&of&at&by&for&with&about\\ \hline 
against&between&into&through&during&before&after\\ \hline 
above&below&to&from&up&down&in\\ \hline 
out&on&off&over&under&again&further\\ \hline 
then&once&here&there&when&where&why\\ \hline 
how&all&any&both&each&few&more\\ \hline 
most&other&some&such&no&nor&not\\ \hline 
only&own&same&so&than&too&very\\ \hline 
s&t&can&will&just&don&don't\\ \hline 
should&should've&now&d&ll&m&o\\ \hline 
re&ve&y&ain&aren&aren't&couldn\\ \hline 
couldn't&didn&didn't&doesn&doesn't&hadn&hadn't\\ \hline 
hasn&hasn't&haven&haven't&isn&isn't&ma\\ \hline 
mightn&mightn't&mustn&mustn't&needn&needn't&shan\\ \hline 
shan't&shouldn&shouldn't&wasn&wasn't&weren&weren't\\ \hline 
won&won't&wouldn&wouldn't\\
\bottomrule
\end{tabular}
\end{center}
\end{table}

In addition to the stop words found in the NLTK library, the words shown in \tableref{tbl:extra_stop_words} were also added to the stop word list. The set of words such as "nypdpsa", "npyd", "nypdct", "seattledot", etc were added to the list to eliminate the bias that those words might introduce during the training. The other words that were added to the stop word list are frequently occurring words such as such as "one", "two", "Sunday", "Monday", "January", "February" whose presence in the text does not affect whether the text is critical or non-critical.

\begin{table}
\begin{center}
\caption{Extra stop words}
\label{tbl:extra_stop_words}
\begin{tabular}{lcccr}
\toprule
nypdprotecting&nypdconnecting&nypd&nypdoneil\\ \hline 
nypdpsa&nypdchiefofdept&nypdbklynnorth&nypdtransit\\ \hline 
nypddetect&nypdnews&nypdspecialops&nypdchiefpatrol\\ \hline 
nypdhighway&nypdtransport&nycsafenypdorg&nypdct\\ \hline 
nypdcentralpark&nycoem&nypddetectives&tcsnycmarathon\\ \hline 
sdnynews&newyorkfbi&nypdoneill&neighborhoodpolicing\\ \hline 
timessquare&victoria&bronx&sunday\\ \hline 
monday&tuesday&wednesday&thursday\\ \hline 
friday&saturday&seattlepd&seattledot\\ \hline 
seattlefir&seattle&brooklyn&manhattan\\ \hline 
one&two&three&four\\ \hline 
five&six&vicpolicenews&victoriapolice\\ \hline 
jointhemet&seattlefir&metpoliceuk&mpswestminster\\ \hline 
london&londonfire&richmond&nycgov\\ \hline 
https&mayorjenny&nyc&nycdot\\ \hline 
january&february&march&april\\ \hline 
may&june&july&august\\ \hline 
september&october&november&december\\ \hline 
joseph&melbourne\\

\bottomrule
\end{tabular}
\end{center}
\end{table}

\subsection{Stemming}
The Porter's algorithm\cite{porter1980algorithm} is a popular stemming algorithm that can reduce the inflected words to their stems. It does so by removing the suffixes from a inflected word using production rules such as the ones shown in \tableref{tbl:porter_algo}. The NLTK python library provides an updated implementation of the Porter's algorithm which can be used for stemming.

\begin{table}
    \centering
    \caption{Porter algorithm stemming rules}
    \label{tbl:porter_algo}
    \begin{tabular}{p{4cm}p{4cm}}
    \toprule
    Rule&Example \\
    \midrule
    (m$>$0) FUL -$>$  &  hopeful -$>$ hope \\
    (m$>$0) ICATE -$>$ IC  & triplicate -$>$ triplic \\
    (*v*) ED -$>$ & plastered -$>$ plaster \\
    \bottomrule
    \end{tabular}
\end{table}

\subsection{IDF Calculation}
The IDF values for the stems are calculated using the equation \eqref{eq:idf_equation}, which are then used to sort the stems in descending order of IDF values. The sorted list for the top 50 stems with their IDF value is shown in \tableref{tbl:idf_stems}. The table shows that the IDF value of stems such as "thief", "alcohol", "tragedi", "food", "trade", "citizen", "civilian", etc is pretty high. The presence of such stems in a text could be a strong indication of whether the text is critical or non-critical.

Since it might not be practical to use every single stem as a feature for the classifier, the number of stems should somehow be reduced. The simplest way to accomplish that task is to select the top $n$ stems for the sorted stem list. For this research, the top 1000 stems were used from the sorted stem list. The complete list of the selected stems is shown in \appendixref{1000_stems}. The list shows that the selected stems does include important stems like "murder", "robberi", "shoot", "burglari", "terror", "death", "victim" which is a desirable outcome.

\begin{table}
\begin{center}
\caption{Top 50 IDF stems}
\label{tbl:idf_stems}
\begin{tabular}{lc|cc|cr|}
\toprule
stem&IDF value&stem&IDF value&stem&IDF value\\
\midrule
graffiti&7.46&ring&7.33&harrow&7.33\\ \hline 
colour&7.33&mother&7.21&food&7.21\\ \hline 
tragedi&7.21&veteran&7.21&bay&7.21\\ \hline 
leg&7.21&tuozzolo&7.21&church&7.21\\ \hline 
design&7.21&termin&7.21&vote&7.21\\ \hline 
mornington&7.21&psos&7.21&brief&7.21\\ \hline 
babi&7.1&race&7.1&pull&7.1\\ \hline 
trade&7.1&post&7.1&coordin&7.1\\ \hline 
class&7.1&extra&7.1&complaint&7.1\\ \hline 
ps&7.1&came&7.1&worth&7.1\\ \hline 
teamwork&7.1&adult&7.1&civilian&7.1\\ \hline 
notic&7.1&collaps&7.1&brown&7.1\\ \hline 
greatest&7.1&terenc&7.1&count&7.1\\ \hline 
medic&7.1&plenti&7.1&akay&7.1\\ \hline 
one&7.1&citizen&7.1&noth&7.1\\ \hline 
strand&7.1&page&7.1&alcohol&7.1\\ \hline 
true&7.1&thief&7.1&&\\
\bottomrule
\end{tabular}
\end{center}
\end{table}

% These stems that were selected are precisely the features that the classifier would use to assign the class to a text.

\subsection{TF-IDF}
Once the IDF values for the terms have been calculated, the TF-IDF algorithm can be used to convert the text into a vector by multiplying the TF of the term with its IDF value as shown in equation \eqref{eq:tf_idf_equation}. The number of dimensions that the TF-IDF vector would have is precisely the number of stems selected as the vocabulary during the IDF calculation.

The vector $v$ obtained is then normalized using the $L^2$ norm as follows,
\begin{equation}
    v_{norm} = \frac{v}{\lVert v \rVert}
\end{equation}

The TF-IDF algorithm was implemented with the help of scikit-learn \cite{scikit-learn} python library which provides an interface for computing both the IDF value for the term and calculating the TF-IDF vector for a document.

% Classfication
\section{Classification}\label{classification}
Several classification algorithms can be used to classify the text such as Rule-Based Classifiers, Neural Networks, Logistic Regression, Random Forest, Decision Trees, K-Nearest Neighbor, SVM\cite{cortes1995support}, etc. 
% However, it has been theoretically and empirically shown that SVMs are very well suited for the text categorization problem\cite{joachims1998text}.
The classifier models were implemented using the scikit-learn python library.

\subsection{Models}
\subsubsection{Logistic Regression}
A logistic regression is a binary classification algorithm that tries to construct a decision boundary as one given in \eqref{eq:regularized_logistic_regression} by minimizing a regularized cost for misclassification using a SGD algorithm.

\begin{equation}
    \label{eq:regularized_logistic_regression}
    h_{\theta}(x) = \frac{1}{1 + {e}^{-{\theta}^{t}. x}}
\end{equation}

where $x$ is the input vector and $\theta$ is the vector containing the parameters of the decision boundary. 

\subsubsection{Support Vector Machine}
Support Vector Machines non-linearly map the input vectors to a very high dimensional feature space and tries to construct a linear decision surface in the new feature space \cite{cortes1995support}. The SVM could also use the kernel trick to construct a non-linear decision surface in the new feature space \cite{cortes1995support}. For the critical news detection task, the Radial Basis Function shown in \eqref{eq:rbf_kernel} was used a kernel.

\begin{equation}
    \label{eq:rbf_kernel}
    K(x, x^{'}) = exp(- \frac{{\lVert x - x^{'} \rVert}^{2}}{2 \sigma^{2}})
\end{equation}
Where, $x$ and $x^{'}$ are the vector inputs, $d$ is the degree of the polynomial to use, and $\sigma$ is the smoothness parameter.

\subsubsection{Random Forests}
The random forests\cite{breiman2001random} is an ensemble method for decision trees\cite{quinlan1986induction} which can be used for both classification and regression analyses. The algorithm performs the prediction task by combining the output from independent tree predictors constructed during the training phase \cite{breiman2001random}. 

\subsubsection{Neural Network}
Neural networks are a networked learning model capable of performing both classification and regression tasks. The model hierarchically constructs a deeper representation of the given input and tries to perform the prediction task on the representation constructed. 

The Neural Network model for the critical news detection is a three layered network with a hidden layer, an input, and an output layer.

\subsection{Training and Testing}
At first, the dataset is randomly separated into training set and test set using a 80-20 split. The models are trained using the training set whereas the test set remains unused during the training. 

For each classifier, the 5 fold cross-validation technique was used to evaluate the performance of the model using a random set of hyperparameters. The set of hyperparamters that achieved the highest performance during the cross-validation were selected. The models were then trained on the whole training set using the hyperparameters selected from the cross-validation step.

In order to evaluate the performance, the trained models were used to predict the class assignment for news articles in the test set. 

\section{Summary}
In summary, this chapter describes the steps that were used to construct and evaluate the classifier models. The chapter explains the process that was used to collect, analyze, and store the data. It then describes the algorithms and the libraries that were used to obtain features from a piece of text, which is followed by the methodology that was used to train and test the classifiers.
 \chapter{Experiments and Results} \label{result}
Once the labelled dataset is ready, the classifier models can be trained on the training set and evaluated using the test set. However, it is better to first visualize the data and then proceed with the training.

1000 dimensional TF-IDF vector can be visualized if the number of features is reduced to 2 or 3. The feature reduction could be done by using various dimensionality reduction algorithms \cite{tenenbaum2000global, roweis2000nonlinear}. The t-SNE\cite{maaten2008visualizing} is an algorithm  is a feature reduction method suited well for the purpose of data visualization. In this thesis, the algorithm takes the TF-IDF vectors obtained from the text processing step and produces a 2D vector representation for each document. The reduced vectors were then plotted using the matplotlib\cite{hunter2007matplotlib} python library.

The dataset visualization indicates that the data does not have significant problems for classification.  Once the classifiers were trained, their performance can be evaluated on the test set.

\section{Dataset Visualization}
\figureref{fig:dataset} shows the result of applying t-SNE algorithm on the dataset. It shows that on average, the feature vectors of significant news are different from the feature vectors of non-significant news which is a desired outcome. However, the dataset does look a bit noisy and the margin of separation between classes is also a bit thin. Such characteristics of the dataset could introduce errors during the classification.

\begin{figure}[h]
    \cfig{4}{data_visualization.png}{5}
    \caption{Dataset visualization using t-SNE}
    \label{fig:dataset}
\end{figure}

\section{Classification Performance}
The performance of the classifiers for five separate runs on the significant news dataset is shown in \tableref{tbl:critical_news_performance} and the average performance of those runs is shown in \tableref{tbl:average_performance}. The results show that on average, the models achieve a reasonably high performance which are comparable to one another; furthermore, it also shows that the neural network model performs slightly better than rest of the models.

\begin{table}
\begin{center}
\caption{Performance of classifiers on significant news dataset}
\label{tbl:critical_news_performance}
\begin{tabular}{lccccr}
\toprule 
Model&Iteration&Accuracy&Precison&Recall&Fscore\\
\midrule 
\multirow{5}{*}{Regularized Logistic Regression}
&1&92.30&91.11&90.56&90.83\\
&2&\textbf{93.93}&90.62&92.38&91.46\\
&3&92.07&89.84&89.84&89.84\\
&4&93.70&92.45&91.23&\textbf{91.82}\\
&5&92.07&92.10&88.94&90.31\\
\hline
\multirow{5}{*}{Random Forests}
&1&90.90&90.54&87.75&88.96\\
&2&89.97&88.56&86.87&87.65\\
&3&\textbf{94.63}&92.42&93.23&\textbf{92.81}\\
&4&91.84&89.94&90.81&90.36\\
&5&92.77&90.34&91.44&90.86\\
\hline
\multirow{5}{*}{SVM}
&1&\textbf{94.06}&93.94&91.24&92.48\\
&2&93.93&92.53&92.92&\textbf{92.72}\\
&3&93.24&92.50&91.09&91.75\\
&4&93.24&91.93&91.73&91.83\\
&5&92.77&91.44&90.34&90.86\\
\hline
\multirow{5}{*}{Neural Network}
&1&94.40&92.21&92.75&92.48\\
&2&\textbf{94.63}&93.74&92.52&\textbf{93.11}\\
&3&92.54&90.03&90.03&90.03\\
&4&93.93&91.69&91.69&91.69\\
&5&92.07&90.06&90.06&90.06\\
\bottomrule
\end{tabular}
\end{center}
\end{table}

\begin{table}
\begin{center}
\caption{Average performance of classifiers on critical news dataset}
\label{tbl:average_performance}
\begin{tabular}{lcccr}
\toprule 
Model&Accuracy&Precison&Recall&Fscore\\
\midrule 
Regularized Logistic Regression&92.81&91.224&90.02&90.852\\
Random Forests&92.022&90.36&90.02&90.128\\
SVM&93.448&92.468&91.464&91.128\\
Neural Network&\textbf{93.654}&91.602&91.685&\textbf{91.43}\\
\bottomrule
\end{tabular}
\end{center}
\end{table}

As shown in \figureref{fig:dataset}, the feature vectors of significant news are on average different from the feature vectors of non-significant news. So, there is a noticeable separation between the two classes. Many classifier models can benefit from this separation. As the separation is even detectable in a reduced feature space, it should be the case that the news or messages could also be classified by using more than 2 features.  This, indeed, seems to be the reason behind the decent performance of the logistic regression model.

Since the logistic regression model was able to achieve a good performance on the test dataset, it would imply that complex classifier models should at least achieve the same performance as the logistic regression model. %However, the problems with the dataset itself would limit the models from obtaining a significantly better performance than the Logistic Regression model. 
The average performance of the models shown in \tableref{tbl:average_performance} seem to comply with that hypothesis.

Some of the correctly classified messages by the neural network model in the test set are shown in \tableref{tbl:correct_predictions}. It shows that the classifier is able to identify the significant news articles when the text clearly contains important stems such as "detect", "robberi", "stab", "fatal", etc. Similarly, the classifier is able to identify a non-significant news article when the text clearly lacks those kind of stems.

\begin{table}
\begin{center}
\caption{Sample of correct predictions}
\label{tbl:correct_predictions}
\begin{tabular}{p{9cm}p{2.5cm}p{2.5cm}}
\toprule 
Text&Label&Predicted\\
\midrule 
Detectives are investigating following a robbery outside a licensed premises in St Albans overnight.More &Significant&Significant\\
\hline
Homicide Squad detectives have charged a man following an alleged fatal stabbing at Ultima.&Significant&Significant\\
\hline
Detectives investigating bank robbery approx. 2:15 pm in 600 blk S. Dearborn.  Suspect fled, at large. He is described as white male, 30s, 5'7", dark bb cap, grey hooded sweatshirt. If you recognize him, contact SPD. https://t.co/Fg4Cblhhxd.&Significant&Significant\\
\hline
RT seabikeblog: Had a bike stolen recently? May want to SeattlePD’s GetYourBikeBack timeline. Just posted a bunch. SEAbikes https://t.c…
&Non-Significant&Non-Significant\\
\hline
There's two weeks left to vote in the Met Excellence Awards. MPSTootingTnC are nominated for Safer Neighbourhoods Team of the Year for their diligent work tackling an increasing drug anti-social behaviour issue in the area. Learn more; cast your vote&Non-Significant&Non-Significant\\
\hline
It was over before it began for this 18-year-old who lost her licence after only two hours&Non-Significant&Non-Significant\\
\bottomrule
\end{tabular}
\end{center}
\end{table}

But the classifier does incorrectly label several texts during the testing, some of which are shown in \tableref{tbl:incorrect_predictions}. The model seems to have a problem of identifying a non-significant news when the text contains stems which are common found in significant news. It could indicate that the TF-IDF vector representation of the text is inadequate to capture the semantics behind it. Similarly, the mislabelling of the significant news articles could have been because of the inability of the classifier to interpret the meaning and implications of the text. One major issue that could lead to incorrect classification is the tokenization. Since tokenization removes numbers from the text, it also eliminates information about date, time and ages. In other words, tokenization loses the temporal context, which could be the reason for the last misclassification example in \tableref{tbl:incorrect_predictions}.

\begin{table}
\begin{center}
\caption{Sample of incorrect predictions}
\label{tbl:incorrect_predictions}
\begin{tabular}{p{9cm}p{2.5cm}p{2.5cm}}
\toprule 
Text&Label&Predicted\\
\midrule 
RT MPSNorthEndRY: We've just found this sword in a bush in comptonplace whilst completing a weapons sweep in partnership with MPSErith…&Significant&Non-critical\\
\hline
Police will remain on the scene at Flinders Street well into the evening and tomorrow morning. We ask the community to continue to avoid the highlighted area. We will provide an update once the crime scene has been cleared. &Significant&Non-Significant\\
\hline
Police Officer Glenn Doss, Jr. of the detroitpolice lost his life protecting the residents of Detriot. NeverForget &Significant&Non-Significant\\
\hline
NeighborhoodPolicing is NYPDprotecting NYPDconnecting: Our NYPD43Pct officers visited Stevenson HS to share information with Bronx students on healthy dating relationships as part of TeenDatingViolenceAwarenessMonth.&Non-Significant&Significant\\
\hline
Victoria Police and VicRoads were on site last night in St Kilda to ensure the safety of motorists and onlookers as the Loy Yang generator (which weighs 650 tonnes and is as large as a jumbo jet!) made its way very slowly to the Melbourne docks. Go team! https://t.co/Ir6qBFlNab
&Non-Significant&Significant\\
\hline
Robert was 23 years old when he got behind the wheel intoxicated, struck two pedestrians and killed one of them. He now shares his personal experience as part of the Cool Heads road safety program. Read more in the Police Life Summer edition → https://t.co/HoBotyjygm https://t.co/3eu9az9GNg&Non-Significant&Significant\\
\bottomrule
\end{tabular}
\end{center}
\end{table}

In a similar manner, the class assignment predicted by the neural network model on some recent news  is shown in \tableref{tbl:real_prediction}. The table shows that the model perform fairly good in the recent news articles. The statements which clearly contain important stems like "beat", "death", "shoot", "police", etc are correctly labelled as significant and articles which clearly do not contain those kinds of features are labelled as non-significant.

\begin{table}
\begin{center}
\caption{Predictions on some real word data}
\label{tbl:real_prediction}
\begin{tabular}{p{12cm}p{2.5cm}}
\toprule 
Text&Predicted\\
\midrule 
A man suspected of viciously beating two homeless men to death in California was previously deported six times. &Significant \\
\hline
A shooting has reportedly occurred at Hebron High School in Carrollton, Texas, according to Fox 4 News.  Police say the scene is now secure, and that one adult was shot during a junior varsity football game. No students were involved. &Significant \\
\hline
A body believed to be that of a 6-year-old boy with special needs who went missing in North Carolina was found in a creek near the park where he was last seen, authorities said.&Significant \\
\hline
At least 832 people now known to have died in Indonesian earthquake and tsunami - emergency officials &Non-Significant \\
\hline
Kanye West doesn’t see any reason why he can’t support Donald Trump and Colin Kaepernick at the same time – and he has the outfit to prove it.  The 41-year-old rapper turned up at the New York office of The Fader rocking both a “redesigned” MAGA hat and a sweatshirt with the controversial football player’s name printed across the chest on Thursday morning.&Non-Significant \\
\hline
CHICAGO (AP) - Following years of reformulating at McDonald's, the company announced that seven of its burgers are now preservative-free. As of Wednesday, the world's largest burger chain the changes have been applied to their: hamburger, cheeseburger, double cheeseburger, McDouble, Quarter Pounder with Cheese, double Quarter Pound with Cheese and Big Mac. &Non-Significant \\
\bottomrule
\end{tabular}
\end{center}
\end{table}

\section{Summary}
This chapter analyzes the performance of various classifiers (logistic regression, random forests, SVM, and neural network model) on the significant news dataset. %Even though the dataset was somewhat noisy, the models performed fairly good. 
On average, the models were able to achieve an accuracy greater than 90 percent with neural network model achieving the highest accuracy of 93.654 percent.
 \chapter{Conclusion}\label{conclusion}
The key idea that this thesis explores is that for fake news detection there are articles on topics such as infotainment, personal news, health tips, etc which are somewhat unimportant so it is not necessary to verify the truthfulness of such articles.

In order to formalize the problem better, a new term "critical news" was defined,

\enquote{A text is labeled as critical if it affects significant number of people, changes the routines of daily life, and needs verification on the information presented.}

The goal of the thesis was then to see if it was possible to construct a classifier that was able to separate a critical news story from a non-critical news story.

A new dataset was created by manually labelling the posts of several Twitter accounts: NYPD News, Metropolitan Police, Victoria Police, Seattle Police Dept, and NYPDCounterterrorism as either critical or as non-critical. The result of that effort was a dataset with 1548 critical and 595 non-critical Tweets.

The TF-IDF algorithm was then used to convert an article to a 1000 dimensional vector that would capture the distinguishing features of the text. The vectors were then used to train and evaluate the performance of multiple classifiers: Regularized Logistic Regression, Random Forests, Support Vector Machine, and Neural Network.

It was observed that the classifiers performed reasonably well on the critical news detection task. The models on average were able to achieve a test accuracy greater than 90 percent with the Neural Network model achieving the high test accuracy of 93.654 percent.

Thus, the results from the experiments indicate that it is possible to construct quite an accurate classifier for separating critical news from non-critical news. Once the classifier is trained, it can be used either as a standalone tool with human reviewers to detect fake news, or it could be used with another automated fake news classifier to filter critical news that should be verified for its truthfulness.

\section{Future Works}
There are several possibilities that could be explored to further improve the performance of the classifier. The first step would be to refine the data and extend it if possible. Adding more samples to the dataset and eliminating noise generally improves the performance of most classifiers, which should be the case here as well.

Another possibility that could be explored is to use a better feature representation of the text. The TF-IDF feature weighing scheme could be extended to other ngram models such as bigram, trigram, etc to obtain the vector representation of the text. Using higher ngram models could provide better representation of the text which could improve the performance. The feature representation of the text could also be improved by using the word embedding models. It has been empirically shown that most of the existing NLP systems can be improved by using word-embedding as an extra feature\cite{turian2010word}. However, the extent of improvement that word-embedding models could bring about for critical news detection remains to be seen.

Yet another possibility that could be explored is replacing the existing classifiers with sequence analysis algorithms such as HMM\cite{baum1966statistical}, LSTM\cite{hochreiter1997long}, or GRU\cite{cho2014learning}. Such algorithms could improve the prediction performance by being able to construct an internal representation of the text that captures its context and meaning.

 
 %   Include my bibliography
\bibliographystyle{unsrt}
%\bibliographystyle{ieeetr}
\begin{singlespace}
\bibliography{references}
\end{singlespace}


%   This set of LaTeX files uses the Appendix Package to handle
%   the appendices - it creates the "Appendices" page and puts
%   "Appendix A:" in the TOC.  So, strictly speaking, the
%   \appendix command is not necessary (because the appendix
%   package uses an "appendices environment").  But, for
%   the typedref package to operate correctly, you must include the
%   \appendix command - so, it's included after the beginning of
%   the appendices environment.  If you only have one appendix,
%   then you need to remove the "page" option from the uahdis.sty
%   file ('page' option occurs when loading appendix package).
%   You can add or subtract appendices - they are treated just like
%   chapters - the only difference is that they occur within the
%   appendices environment, which tells LaTeX to give them letters
%   instead of numbers.
\begin{appendices}
\appendix
%
%%   Include my appendices

\chapter{List of 1000 stems with highest IDF values}\label{1000_stems}

\begin{longtable}{p{2.5cm}p{2.5cm}p{2.5cm}p{2.5cm}p{2.5cm}}
graffiti&ring&harrow&colour&mother\\ \hline 
food&tragedi&veteran&bay&leg\\ \hline 
tuozzolo&church&design&termin&vote\\ \hline 
mornington&psos&brief&babi&race\\ \hline 
pull&trade&post&coordin&class\\ \hline 
extra&complaint&ps&came&worth\\ \hline 
teamwork&adult&civilian&notic&collaps\\ \hline 
brown&greatest&terenc&count&medic\\ \hline 
plenti&akay&one&citizen&noth\\ \hline 
strand&page&alcohol&true&thief\\ \hline 
encourag&foxnew&yrs&product&chanc\\ \hline 
tunnel&pride&ali&green&woolwich\\ \hline 
town&mpsonthestreet&fraudster&chase&african\\ \hline 
vicpol&ashton&creek&hume&dandenong\\ \hline 
limit&kmh&amaz&american&felon\\ \hline 
confirm&intellig&newli&spent&saw\\ \hline 
wait&entir&hold&mourn&loss\\ \hline 
pay&beauti&presid&southern&blvd\\ \hline 
equip&record&agenc&hundr&ann\\ \hline 
real&peac&protector&shirt&countri\\ \hline 
borough&test&bicyclist&main&foot\\ \hline 
opportun&ullah&visitor&late&modern\\ \hline 
frequent&feet&backpack&websit&cab\\ \hline 
happen&cours&steal&weve&talktom\\ \hline 
intel&miller&familia&balloon&base\\ \hline 
pursuit&whereabout&mope&syria&georg\\ \hline 
mill&southwark&magistr&surround&height\\ \hline 
ballarat&motor&hitrun&melton&enrout\\ \hline 
fbi&sincer&iraq&fidelisadmortem&enter\\ \hline 
luck&listen&abc&recogn&rais\\ \hline 
spoke&awar&approx&handgun&storm\\ \hline 
warn&power&firefight&prioriti&brighton\\ \hline 
small&especi&pass&america&explain\\ \hline 
preliminari&treat&staff&pant&remov\\ \hline 
hair&agent&trust&welfar&ever\\ \hline 
extend&bombcyclon&extrem&entri&tragic\\ \hline 
santa&navi&aircraft&trooper&word\\ \hline 
outstand&rockcenterxma&passeng&maci&kensington\\ \hline 
brent&barnet&raid&bark&daesh\\ \hline 
hounslow&war&club&cressida&jackson\\ \hline 
taskforc&down&leak&must&univers\\ \hline 
period&werribe&air&stole&caution\\ \hline 
suburb&ballard&thousand&apart&forcibl\\ \hline 
led&vigil&said&crowd&histori\\ \hline 
headquart&load&resourc&grand&broadway\\ \hline 
credit&fraud&neighbor&crimin&materi\\ \hline 
suffer&counti&thing&within&inspector\\ \hline 
approxim&answer&usarmi&catch&matter\\ \hline 
christoph&toward&human&technolog&there\\ \hline 
offici&advic&author&begin&congrat\\ \hline 
merri&hudson&particip&challeng&deceas\\ \hline 
haringey&keen&thiev&offend&plead\\ \hline 
speed&boat&kilda&impound&capitol\\ \hline 
resolutesupport&past&byrn&account&order\\ \hline 
win&spend&case&excel&smile\\ \hline 
firework&stand&execut&traffick&social\\ \hline 
risk&william&jacket&birthday&target\\ \hline 
role&global&pic&behaviour&brave\\ \hline 
intern&fail&lake&ne&militari\\ \hline 
condit&date&beyond&direct&hq\\ \hline 
reduct&spread&vow&minut&reunit\\ \hline 
feel&mcdonald&bodi&nov&alon\\ \hline 
play&mark&turn&motorist&delay\\ \hline 
twitter&dark&river&owner&portauthor\\ \hline 
feder&mind&sister&convict&recognis\\ \hline 
stopknifecrim&dick&tram&sunshin&display\\ \hline 
engag&gather&sleep&brother&packag\\ \hline 
touch&safer&battl&heart&cheer\\ \hline 
initi&colleagu&game&alert&camera\\ \hline 
done&cross&monahan&got&finest\\ \hline 
donat&approach&daniel&king&lambeth\\ \hline 
onthisday&third&sw&hrs&ft\\ \hline 
sweep&gas&hispan&appreci&here\\ \hline 
domest&center&mean&sad&parti\\ \hline 
transit&item&much&partnership&question\\ \hline 
littl&mr&fellow&elder&david\\ \hline 
remind&import&crisi&messag&po\\ \hline 
everyday&away&maleblack&major&track\\ \hline 
warm&behind&lot&statement&mount\\ \hline 
highway&enfield&ilford&mobil&islington\\ \hline 
burglar&set&lifethreaten&flinder&afghanistan\\ \hline 
bank&leader&card&end&drop\\ \hline 
whether&gang&recoveri&cash&put\\ \hline 
histor&wound&tri&differ&canin\\ \hline 
roll&deploy&award&site&fox\\ \hline 
tell&facebook&andrew&flyover&increas\\ \hline 
motorcycl&croydon&whilst&greenwich&van\\ \hline 
stream&cbd&blk&rainier&seaneighborhood\\ \hline 
nd&kid&reward&abus&winter\\ \hline 
threaten&academi&downtown&sergeant&age\\ \hline 
low&cold&hall&personnel&steven\\ \hline 
dementia&retir&nyer&happynewyear&festiv\\ \hline 
gift&caus&port&helicopt&footag\\ \hline 
sought&walkthemet&aurora&defeatdaesh&relationship\\ \hline 
transport&sign&tune&squar&lower\\ \hline 
sever&properti&statenisland&present&prevent\\ \hline 
uniform&youth&second&forward&rob\\ \hline 
australia&offer&your&counter&beach\\ \hline 
camden&nb&keepyourcool&exampl&deliv\\ \hline 
terrorist&safest&along&eye&spot\\ \hline 
snow&attend&prison&counterterror&host\\ \hline 
white&rd&effect&million&forget\\ \hline 
proud&det&central&long&jame\\ \hline 
known&newham&cctv&stopper&hmc\\ \hline 
front&hand&ride&leav&better\\ \hline 
annual&recruit&weather&hit&receiv\\ \hline 
paul&think&clear&midtown&phone\\ \hline 
develop&reopen&coalit&westminst&violent\\ \hline 
hackney&summer&aggrav&altern&address\\ \hline 
student&hear&recent&danger&possess\\ \hline 
drink&welcom&rape&precinct&point\\ \hline 
hour&someon&other&visionzero&chang\\ \hline 
fli&pc&caught&sb&seattlefir\\ \hline 
lbs&ceremoni&send&seek&dog\\ \hline 
walk&seiz&abl&water&photo\\ \hline 
program&presenc&control&tree&outsid\\ \hline 
could&deal&effort&parad&tackl\\ \hline 
carmenbest&readi&child&group&vicin\\ \hline 
care&prepar&commit&latest&retweet\\ \hline 
season&soon&avenu&later&big\\ \hline 
law&money&love&bus&concern\\ \hline 
open&avoid&threat&ongo&monitor\\ \hline 
occur&district&carri&talk&quick\\ \hline 
blue&never&return&lead&bike\\ \hline 
struck&suspici&fun&run&rescu\\ \hline 
overnight&advis&let&next&possibl\\ \hline 
dont&action&ask&onlin&link\\ \hline 
centr&moment&store&graduat&right\\ \hline 
john&affect&urg&imag&individu\\ \hline 
happeningsoon&give&custodi&condol&full\\ \hline 
advisori&took&yo&senior&anoth\\ \hline 
stori&young&truck&rout&constabl\\ \hline 
tomorrow&high&school&hard&show\\ \hline 
would&short&bridg&critic&sacrific\\ \hline 
bomb&learn&dec&face&sgt\\ \hline 
becom&knife&special&sentenc&activ\\ \hline 
trace&subway&fallen&children&lost\\ \hline 
boy&busi&discuss&via&promot\\ \hline 
eve&appear&lane&violenc&still\\ \hline 
save&state&hous&larg&closur\\ \hline 
court&seri&feb&due&nycsaf\\ \hline 
guilti&weapon&first&announc&relat\\ \hline 
conduct&memori&act&left&enforc\\ \hline 
taken&result&fdni&forc&remain\\ \hline 
launch&ensur&deputi&believ&video\\ \hline 
ahead&sure&best&place&current\\ \hline 
spd&warrant&pedestrian&travel&thanksgiv\\ \hline 
macysparad&ago&old&girl&explos\\ \hline 
isi&multipl&icymi&depart&light\\ \hline 
provid&teen&confer&resid&check\\ \hline 
hero&bring&job&femal&identifi\\ \hline 
firearm&addit&crash&dedic&involv\\ \hline 
togeth&east&wish&world&anyon\\ \hline 
shop&serv&wear&expect&throughout\\ \hline 
back&head&nycmayor&number&hill\\ \hline 
attempt&issu&contact&releas&sexual\\ \hline 
friend&stay&sinc&christma&block\\ \hline 
duti&happeningnow&partner&start&south\\ \hline 
alway&york&regard&jan&hope\\ \hline 
offenc&line&meet&yorker&emerg\\ \hline 
nation&build&name&theft&fight\\ \hline 
prayer&media&squad&celebr&everi\\ \hline 
even&across&afternoon&serious&hospit\\ \hline 
command&also&thought&homicid&west\\ \hline 
drug&death&newyearsev&patrol&stolen\\ \hline 
go&around&visit&avail&station\\ \hline 
secur&ny&like&week&holiday\\ \hline 
oper&read&injur&recov&earli\\ \hline 
neighborhood&earlier&injuri&come&north\\ \hline 
congratul&includ&someth&teenag&id\\ \hline 
weekend&know&met&insid&mani\\ \hline 
way&speak&alleg&event&need\\ \hline 
life&connect&close&plan&driver\\ \hline 
respond&queen&terror&part&drive\\ \hline 
search&support&happi&scene&fire\\ \hline 
black&person&illeg&stab&train\\ \hline 
arm&die&burglari&honor&attack\\ \hline 
month&home&local&respons&enjoy\\ \hline 
tonight&report&traffic&stop&commission\\ \hline 
say&team&safeti&shot&made\\ \hline 
rememb&cop&protect&vehicl&wit\\ \hline 
gun&everyon&good&look&jail\\ \hline 
member&chief&unit&fatal&kill\\ \hline 
watch&use&found&servic&women\\ \hline 
near&park&th&find&make\\ \hline 
night&time&join&yesterday&seen\\ \hline 
car&victim&shoot&road&continu\\ \hline 
live&ave&take&get&assault\\ \hline 
famili&area&news&male&detail\\ \hline 
robberi&citi&well&suspect&incid\\ \hline 
pleas&collis&pm&men&neverforget\\ \hline 
peopl&woman&day&charg&st\\ \hline 
communiti&murder&updat&see&street\\ \hline 
public&tip&assist&want&info\\ \hline 
morn&call&work&crime&yearold\\ \hline 
great&new&last&inform&locat\\ \hline 
share&keep&pct&us&thank\\ \hline 
miss&year&appeal&arrest&detect\\ \hline 
today&follow&safe&investig&help\\ \hline 
amp&man&offic&rt&polic\\ \hline 
\end{longtable}


%%\include{./CH8/appendix2name}
%%\include{./CH9/appendix3name}
%
%%   Here comes the stuff after the main content
%%   turn off appendix environment/package first:
\end{appendices}

\backmatter

%   Include my biography
%\include{./BACK/biography}
% MWT - biography not included in UAH thesis

%   We're done.
\end{document}
