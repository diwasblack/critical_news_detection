\chapter{Background}
\label{ch:background}

Protein crystallization is the process of growing large protein crystals in a solution to identify the protein structure. The solution typically contains multiple reagents such as a buffer, precipitants, and salts. These factors affect the outcome of protein crystallization. Setting up an experiment using various combination of these reagents is called screening. The total number of possible combinations of these reagents is so large that it is practically impossible to try all of them. Moreover, due to the high expenses of carrying out these experiments, to exhaustively try out numerous combinations in the wet lab is not an economical option. Commercial screens with a predefined set of conditions are available to help researchers select suitable conditions that can produce crystals. However, the number of these predefined set of conditions is limited. Furthermore, the conditions present in the commercial screens are not guaranteed to crystallize difficult proteins\cite{SamyamThesis}. Although the number of conditions of these commercial screens is limited, a more thorough analysis of these conditions might yield new conditions that can produce large crystals.

\section{Related Work} \label{sec:bg-related}

Protein crystallization screening methods fall into one of two categories depending on whether or not they utilize the results of previous experiments\cite{Montgomery}. The methods of the first category typically analyze balanced distribution of reagents instead of looking at the results of previous experiments. Full Factorial Design and Incomplete Factorial Design are some of such methods. While full factorial design is an exhaustive and practically infeasible way of testing all possible combinations of reagents, incomplete factorial design \cite{carter1979protein} focuses on interactions among reagents by setting balanced distribution of reagents in the experiments.  Moreover, in incomplete factorial method, the number of experiments to be carried out is significantly reduced by using statistical methods to figure out the important factors for crystallization. The Sparse Matrix Sampling \cite{jancarik1991sparse} method utilizes a sparse matrix of trial conditions chosen from previous crystalline conditions.
%favor exhaustive approach and use all combinations of reagents and concentrations to generate a list of candidate cocktails.

So the question is, what can be done if the results obtained from the available screens are not satisfactory? The methods in the second category directly address this question by employing some heuristics to generate candidate cocktails based on the results of previous experiments. Regression, classification, association analysis have been used for this purpose\cite{DataMining1, DataMining2}. However, these methods were not efficient because of insufficient and highly skewed data, and a very large solution space\cite{DataMining3}.

Associative experimental design (AED) \cite{OptimizeAED,dinc_protein_2015} and GenScreen  \cite{SamyamThesis} have successfully generated crystalline conditions by analyzing the results of previous experiments. 
AED, proposed by Dinc et al., identifies screening factors that are more likely to produce crystalline conditions. It searches for conditions that have common reagents and then swaps them with other reagents to generate new conditions. 
The use of genetic algorithm for protein crystallization screening was first put forward by Saridakis  \cite{Emmanuel}, which used the principles of crossover and mutation for 2 generations to produce new conditions based on previously tested crystalline conditions. GenScreen, proposed by Acharya, assigns scores to reagents and employs genetic algorithm for hundreds of iterations to produce novel conditions\cite{SamyamThesis}.

While  AED generates novel conditions, the output conditions are somewhat similar to the original conditions since the parents of new conditions share a common reagent. GenScreen improves search space using its randomized feature of selecting parents and mutations. However, the chemical space is still not necessarily explored effectively because GenScreen is inherently limited by its objective based fitness function which can prematurely converge to locally optimum solutions.

The lack of exploratory component in these algorithms demands a novel algorithm that can explore the chemical space efficiently. This is because even the discovery of one novel crystalline condition is a significant contribution to the application field of protein crystallization. Visualization of protein crystals has a significant importance in clinical chemistry and has, for well over two decades, made it possible for researchers to generate new chemical ideas for drug discovery \cite{blundell2004high, williams2005recent, sharff2003high, kuhn2002genesis, hardy2003impact}.

Novelty search, developed by Joel Lehman and Kenneth O. Stanley \cite{Novelty}, is one such algorithm that explores a large area of search space. It is a variation of the genetic algorithm that pushes the population away from previously discovered areas, thereby discovering novelty. This is achieved by tuning the fitness function such that it directly searches for novelty. One advantage of using a \textit{divergent} algorithm like novelty search is that by directly searching for novelty, it avoids premature convergence\cite{novelty_premature_convergence}.

Novelty search can be implemented on top of most evolutionary algorithms\cite{Novelty}. It has demonstrated its effectiveness over traditional objective-based search and random search\cite{novelty_premature_convergence} in the deceptive robot maze navigation task\cite{Mouret2011} and the artificial ant benchmark task\cite{Koza_1992}.

While novelty search can explore search space effectively, to the best of our knowledge, no effort has been made to reduce the overhead that lies in the calculation of the novelty metric, which is central to any implementation of the novelty search.
We would like to demonstrate a way to reduce the overhead that lies in the redundant calculation of the novelty score or more commonly known in the domain of genetic algorithm as the fitness score.
Furthermore, we would also like to show that by using novelty search, additional distinct crystalline conditions not detected by AED and GenScreen could be generated. %while also maintaining a significant overlap with those methods.


\section{Setting Up Experiments and Scoring} \label{sec:bg-setting}

\begin{figure}[H]
	\cfig{2}{screen.jpg}{5}
	\caption{Sample Plate for Crystallization Screening}
	\label{fig:plate}
\end{figure}

Even when computational methods are technically sound, they still have to be evaluated in a wet lab to see the actual outcome of these methods. Setting up experiments is important for the evaluation of our method.

Protein crystals are formed in a saturated solution in the wet lab. This solution is made up of crystallization reagents such as buffers, precipitants, salts, and other additives\cite{SamyamThesis}. The buffers are used to maintain the pH of the solution and additives are used to control the solubility of a protein. These solutions are then filled separately in different wells of a plate(\figureref{fig:plate}) and typically, the experiment is observed for a few weeks. The wells are periodically observed from a microscope or a high-throughput system to check if any crystals have formed\cite{microscope}. Crystals are successfully formed only when a specific activation energy and an ordered sequence of inter-molecular-interactions are present \cite{SamyamThesis,ActivationEnergy,InterMolecular}. We have used the Crystal X2 microscope \cite{sigdel:2013} that was developed at iXpressGenes, Inc to capture images.


%\section{Scoring Protein Crystals} \label{sec:bg-scoring}
In this thesis, we have used revised Hampton scoring\cite{Hampton} as a scoring scheme for protein crystallization outcomes \cite{dinc_protein_2015}. Proteins are scored in the range of 1 to 9 to measure the degree of crystallization, 9 denoting large 3D crystals and 1 denoting clear solution. \tableref{tbl:hamptonScores} \cite{OptimizeAED} shows the list of revised Hampton scores and \figureref{fig:sample-scoreImage} shows images of proteins corresponding to each Hampton score. 
\begin{table}[h]
	\centering
    \caption{List of Hampton and revised scores}
    \label{tbl:hamptonScores}
    \begin{tabular} {|c| c| l |}
    \hline
    \textbf{Hampton scoring} & \textbf{Revised} & \textbf{Outcome} \\
    \hline
	 & 0 & Heavy amorphous precipitate \\
     \hline
     1 & 1 & Clear solution \\
     \hline
     2 & 2 & Phase change (oiling out) \\
     \hline
     3 & 3 & Precipitate (light) \\
     \hline
     4 & 4 & Bright spots or granular precipitate \\
     \hline
     5 & 5 & Spheroids, dendrites, urchins \\
     \hline
     6 & 6 & 1D needles \\
     \hline
     7 & 7 & 2D needles \\
     \hline
     8 & 8 & 3D crystals, small, \textless 200$\mu m$ \\
     \hline
     9 & 9 & 3D crystals, large, \textgreater 200$\mu m$ \\
     \hline
    \end{tabular}
\end{table}

\newpage
\begin{figure}[H]
	\cfig{2}{ScoreImage.jpg}{3.5}
	\caption{Sample microscopic images of the protein crystallization outcomes. (0) 	Heavy amorphous precipitate (1) Clear solution (2) Phase change (oiling out) (3) 	 Precipitate (light) (4) Bright spots or granular precipitate (5) Spheroids, 		dendrites, urchins 
	(6) 1D needles (7) 2D plates (8) 3D crystals, small (9) 3D crystals, large}
	\label{fig:sample-scoreImage}
\end{figure}



\section{Summary} \label{rel:summary}

In this chapter, we provided a brief background on protein crystallization screening process. 
%We described the analysis of experiments based on commercial screens tested in a wet lab. 
We also presented some of the related work done in the field of protein crystallization screening, genetic algorithm, and novelty search. %We then explained that the the only way of measuring the actual performance of our algorithm is to test the conditions generated in a wet lab.


